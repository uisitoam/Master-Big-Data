% filepath: /home/uisito/Documentos/Master-Big-Data/Aplicaciones y casos de uso empresarial/tu_archivo.tex
\documentclass[twoside, 12pt]{opticajnl}
\journal{opticajournal} % use for journal or Optica Open submissions

% See template introduction for guidance on setting shortarticle option
\setboolean{shortarticle}{true}
% true = letter/tutorial
% false = research/review article

\usepackage[backend=biber, style=apa, natbib=true, maxcitenames=2, maxbibnames=20, minbibnames=1]{biblatex} 
\addbibresource{references.bib} % filename of the bibliography
\usepackage[autostyle=true]{csquotes} % generate language-dependent quotes in the bibliography

\DeclareCiteCommand{\fcite}
  {\usebibmacro{prenote}\printtext{(}}
  {%
    \bibhyperref{%
      \usebibmacro{citeindex}%
      \printnames{labelname}%
      \setunit{\addcomma\space}%
      \printfield{year}%
      % Uncomment the next line if your style uses extrayear (e.g., 2023a, 2023b)
      %\printfield{extrayear}%
    }%
  }
  {\multicitedelim}
  {\printtext{)}\usebibmacro{postnote}}

\usepackage{lineno}
%\linenumbers % Turn off line numbering for Optica Open preprint submissions.

\title{\textit{Edge computing} en salud}

\author[1,2,3]{Luis Ardévol Mesa}

\begin{abstract}
En este trabajo se presenta una introducción a la computación en el borde (\textit{edge computing}) y sus aplicaciones en el ámbito sanitario. Para ello, se hace una revisión de las aplicaciones generales y se particulariza para el caso de la monitorización de pacientes, donde es necesario introducir el concepto de Internet de las Cosas Médicas. Se muestra la combinación de este último con la computación en el borde mediante ejemplos de monitorización de pacientes crónicos y telemedicina. Tras esto, se comenta brevemente el marco legal que envuelve a los datos clínicos y cómo encaja aquí este paradigma computacional. Se termina exponiendo los retos a los que se enfrenta \textit{edge computing} actualmente y las posibles líneas de innovación que se están explorando. 
\end{abstract}

\setboolean{displaycopyright}{false} % Do not include copyright or licensing information in submission.

\begin{document}

\maketitle

\begingroup
  \hypersetup{linkcolor=black}
  \tableofcontents
\endgroup

\newpage 
\section{Introducción}

Hasta ahora, el \textit{cloud computing} ha sido la opción de gestión y procesamiento de datos por excelencia, ofreciendo servicios bajo demanda y concentrando el procesamiento en grandes centros de datos \fcite{Alshareef2023}. Sin embargo, con la creciente cantidad de datos, el \textit{edge computing} (EC) surge como un nuevo enfoque que traslada parte de la capacidad de cómputo y almacenamiento hacia el borde de la red, más cerca del usuario o de la fuente de datos \fcite{Carvalho2021}. Al reducir la distancia entre generación y procesamiento, se disminuye la latencia, se incrementa la disponibilidad y se aligera el tráfico en los servidores centrales, algo crucial en aplicaciones que requieren decisiones en tiempo (casi) real \fcite{Khan2019}. Por todo lo anterior, el EC se caracteriza por su baja latencia, ancho de banda elevado y acceso inmediato a la información de la red \fcite{Bajic2019, Khan2019}. \\

Aunque la nube ha demostrado ser eficaz gracias a su escalabilidad y facilidad de acceso, el aumento constante de dispositivos conectados y el volumen de datos generados ponen en evidencia sus limitaciones. En entornos dependientes de la latencia o con conexiones irregulares (por ejemplo, en zonas rurales), la capacidad de respuesta de un modelo centralizado puede quedarse corta \fcite{Sadiku2014}. \\

El \textit{edge computing} complementa \textit{cloud computing}, dando lugar a arquitecturas híbridas. En ellas, las tareas que demandan respuesta inmediata se ejecutan en el borde, mientras que las operaciones como el entrenamiento de modelos avanzados siguen recurriendo a la nube. Esto resulta especialmente útil en salud, donde la privacidad, la disponibilidad y la precisión son requisitos fundamentales \fcite{Dave2021}. \\

El objetivo de este trabajo es analizar el papel del \textit{edge computing} en el sector sanitario, con un enfoque especial en la monitorización remota de pacientes. A lo largo del documento expondremos sus ventajas, limitaciones y los desafíos que aún deben afrontarse, incluyendo nuevas tecnologías como biosensores implantables o tatuajes de grafeno.

\section{Aplicaciones generales}

Como se comentó en la introducción, el \textit{edge computing} reduce la dependencia de infraestructuras centralizadas y facilita el procesamiento en tiempo real, lo que lo hace idóneo en diversos sectores \fcite{Khan2019, 9063670}:
\begin{itemize}
    \item Industria: el EC permite un mejor mantenimiento de los equipos, ya que gracias al procesamiento en tiempo real se puede llevar a cabo lo que se conoce como ``mantenimiento predictivo'', es decir, detección de anomalías en equipos antes de un error crítico. 
    \item Transporte: en vehículos autónomos, la capacidad de procesar datos al instante es de gran importancia para tomar decisiones seguras en instantes.
    \item Ciudades inteligentes: desde la gestión del tráfico hasta la seguridad pública, el procesamiento local de datos mejora la eficiencia y la resiliencia de los servicios urbanos.
\end{itemize}

\subsection{\textit{Edge computing} en salud} \label{subsec2a}

El ámbito sanitario aprovecha el EC especialmente en la atención médica. Al procesar datos clínicos localmente, se agilizan los diagnósticos y tratamientos en emergencias, y se facilita la monitorización remota de pacientes. Esta inmediatez permite intervenciones rápidas cuando se necesita hospitalización y aporta mayor precisión en el diagnóstico, apoyando al profesional en la toma de decisiones \fcite{Rancea2024}. \\

La privacidad de los datos médicos es un aspecto clave (que comentaremos más adelante) \fcite{Singh2021}. En España, la Ley Orgánica 3/2018 de Protección de Datos y, a nivel europeo, el RGPD y el Reglamento EEDS, regulan su protección. El procesamiento en el borde minimiza riesgos y facilita el cumplimiento de estas normativas. \\

Aunque la utilidad del EC en la mejora de la calidad del cuidado es obvia, también resulta útil dentro de los hospitales, por ejemplo, al detectar errores en máquinas críticas como los aceleradores lineales usados en radioterapia.

\section{Monitorización de pacientes}

\subsection{\textit{Edge computing} como solución}

La monitorización remota de pacientes resulta una práctica fundamental en la atención sanitaria, basada en dispositivos de IoT que permiten recopilar signos vitales y datos de salud en tiempo real fuera del hospital \fcite{Osama2023, Ghadi2024}. Este enfoque mejora los resultados clínicos al permitir un seguimiento continuo del paciente. \\

En entornos de baja señal o conexión inestable, la dependencia de la nube puede hacer que dar una respuesta inmediata ante una emergencia sea complicado. Además, los dispositivos de monitorización generan un gran volumen de datos, de modo que el ancho de banda resulta un cuello de botella y se pueden producir retrasos en la recepción de información crítica. Para atender todos estos problemas, es necesario optimizar la arquitectura de los sistemas de monitorización remota. \\

Ya se vio como \textit{edge computing} superaba estas limitaciones en un contexto general. De cara a la monitorización de pacientes, el uso de EC implica la posibilidad de procesar los datos clínicos en tiempo real y enviar solo los datos relevantes o alertas médicas, lo que supone un alivio de la carga de la red y optimiza el uso del ancho de banda. Además, la baja latencia característica de este tipo de computación facilita la respuesta inmediata ante emergencias. Otro de los puntos importantes antes mencionados es la conexión de red deficiente; esto con la computación en el borde supone solo un problema parcial: los dispositivos de monitorización son capaces de seguir funcionando y almacenando datos, pero el envío de los mismos en caso de alerta puede verse retrasado \fcite{Osama2023}.  


\subsection{Internet de las Cosas Médicas}

Hoy en día, cuando hablamos de monitorizar pacientes a distancia, no podemos hacerlo sin el Internet de las Cosas Médicas (IoMT). Con él, disponemos de una red en la que distintos dispositivos y aplicaciones clínicas comparten datos en tiempo real para ayudar al personal sanitario a tomar decisiones más informadas \fcite{Osama2023}. Entre estos dispositivos destacan:
\begin{itemize}
    \item \textit{Wearables} sencillos: pulseras o parches que miden constantes vitales básicas, como frecuencia cardíaca o glucosa, y, en modelos más avanzados, llegan a registrar electrocardiogramas completos \fcite{Ghazizadeh2024}.
    \item Biosensores avanzados: equipos implantables o dérmicos que recogen métricas desde dentro del cuerpo, como marcapasos, sensores subcutáneos de glucosa intersticial o tatuajes de grafeno que detectan impulsos eléctricos, hidratación y temperatura, siendo mínimamente invasivos \fcite{Kireev2022}.
\end{itemize}

Todos estos nodos IoMT adquieren información clínica de forma constante, lo que permite vigilar el estado de salud de un paciente las 24 horas del día y avanzar hacia una atención cada vez más personalizada. \\

No obstante, para que este sistema sea realmente útil, es indispensable incorporar \textit{edge computing} lo más cerca posible del paciente. De nada sirve capturar datos en tiempo real si luego llegan con retraso o saturan la red. Hoy en día, un nodo local es capaz de procesar y filtrar la información que recoge, e incluso aplicar modelos de inteligencia artificial, sin necesidad de enviar a la nube \fcite{Osama2023}. Gracias a ello, podemos detectar picos de glucosa al instante con bombas de insulina inteligentes o lanzar alertas automáticas si una persona mayor sufre una caída, todo ello sin depender de una conexión estable.   



\subsection{Casos prácticos}

\subsubsection{Pacientes crónicos}

Antes comentábamos la gran ventaja que supone la computación en el borde, junto con los dispositivos de IoMT para la atención a pacientes con enfermedades crónicas. Por ejemplo, en el manejo de la diabetes, que requiere reacciones casi inmediatas ante variaciones drásticas del azúcar en sangre, los dispositivos suelen valerse de la computación en el borde para procesar datos cerca del paciente \fcite{Klonoff2017}. \\

Un ejemplo real de aplicación de \textit{edge computing} en el seguimiento de pacientes crónicos lo encontramos en el dispositivo iGLU 2.0, un \textit{wearable} no invasivo diseñado para la medición continua de glucosa sérica (presente en el suero de la sangre) \fcite{Joshi2020}. Este sistema emite luz en el infrarrojo cercano (NIR) en 940 y 1300 nm y recoge señales de absorción y reflectancia, lo que permite inferir la concentración de glucosa. \\

El nodo del borde, que en este caso se trata (generalmente) de un \textit{smartphone}, ejecuta en tiempo real un modelo de regresión polinómica o una red neuronal profunda (previamente calibrados), y en base a ello es capaz de generar alertas ante previsión de hipoglucemia o hiperglucemia sin depender de conexión constante a la nube. Los datos ya validados se sincronizan con servidores remotos, de modo que médicos y cuidadores tengan acceso al histórico de datos y puedan realizar análisis más avanzados. \\

Durante la validación clínica, iGLU 2.0 fue probado en sujetos sanos, prediabéticos y diabéticos de 17 a 80 años, resultando adecuado para uso diario y vigilancia continua en el hogar. \\

Otro ejemplo en pacientes con enfermedades crónicas lo podemos ver en \fcite{Matsumura2024}, donde se presenta un parche multimodal para monitorización continua. Los datos son recogidos y convertidos a señales digitales y es el \textit{smartphone} el que actúa como nodo, procesando la información en tiempo real mediante un algoritmo de \textit{reservoir computing} (RNN con capas congeladas, solo entrena la capa de salida lineal) disponible en la aplicación móvil. Este algoritmo clasifica al instante anomalías críticas sin depender de la nube y (opcionalmente) avisa al equipo médico. \\

Metiéndonos un poco más en detalle, el ECG detecta los latidos, por lo que calculando los intervalos RR se puede obtener la frecuencia cardíaca; la respiración se obtiene a partir de las deformaciones del sensor, y la temperatura y humedad se registran de forma continua. \\

Durante las pruebas con voluntarios, el sistema alcanzó una alta precisión en la detección de arritmias, con un retardo en el análisis varios órdenes de magnitud por debajo del segundo, lo que demuestra su viabilidad para un seguimiento domiciliario seguro y continuo.


\subsubsection{Telemedicina}

Otra de las ventajas del uso de \textit{edge computing} es que, al ``independizarse'' de la nube para ciertas actividades, es útil en zonas de baja latencia y/o conexión irregular. Si además tenemos en cuenta los dispositivos IoMT, vemos que la atención a pacientes en zonas rurales es algo que puede beneficiarse enormemente de este nuevo paradigma. Al incorporar nodos \textit{edge} en estos entornos, resulta posible atender de forma adecuada a los pacientes, en especial a personas mayores. En este contexto, se habla de la telemedicina como una mejora del acceso a la atención sanitaria. \\

Como ejemplo de una implementación de computación en el borde en dispositivos IoMT en zonas rurales tenemos el marco desarrollado por \fcite{Prabhu2022}, que se centra en la monitorización remota de la presión arterial. En este \textit{framework}, el \textit{gateway edge} es una Raspberry Pi 4 y, por \textit{bluetooth}, tiene conectado un sensor óptico de presión sanguínea. \\

En el nodo \textit{edge}, las lecturas de presión sistólica, diastólica y frecuencia cardíaca se procesan en tiempo real para generar alertas instantáneas para el personal médico local. En este proceso no existe dependencia de una conexión regular o constante con la nube y solo los eventos relevantes/resúmenes se llevan a la nube cuando la conexión lo permite. Esto, por supuesto, reduce el tráfico de datos en gran medida, demostrando otra vez una de las grandes ventajas del uso de la computación en el borde. \\

Otro ejemplo del trabajo conjunto de los dispositivos IoMT con \textit{edge computing} en telemedicina lo vemos en \fcite{Wang2022}. Usando un \textit{Apple Watch Serie} 6 capturan señales de ECG, frecuencia cardíaca y saturación de oxígeno; estos datos llegan a un nodo MEC basado en Raspberry Pi, donde se analiza la información disponible para detectar distintos tipos de arritmias de forma casi inmediata. Como en el ejemplo anterior, solo las alertas y los resúmenes se sincronizan con la nube. 


\section{Privacidad y gobernanza del dato médico}

Los datos suponen una fuente de información constante y de mucho valor, aunque hay que tener en cuenta la complejidad de gestión, manejo y procesado de los mismos \fcite{Chen2014, Khan2014}. Si hablamos de datos médicos, hay que añadir la protección de la información sensible y el control del paciente sobre el acceso y uso de su información clínica por terceros. Además, también existen estrictas políticas y normativas para garantizar un manejo seguro y ético de los datos clínicos de cada paciente; esto es lo que conocemos como gobernanza. \\

En la sección 2.\ref{subsec2a} introdujimos la privacidad del dato médico como un aspecto clave. En España, la privacidad de estos datos viene protegida por la \cite{BOE}. Además, a nivel europeo existe el \cite{EEDS}, por el cual se establece un marco común para el uso y el intercambio de datos de salud electrónicos en toda la UE. En países como Estados Unidos, estos datos están sujetos a la \textit{Health Insurance Portability and Accountability Act} (HIPAA) de 1996. \\

Como los datos médicos deben cumplir las regulaciones pertinentes tanto en el país de generación de datos como en el de almacenamiento (aunque en general este sea el mismo), el entorno tradicional de \textit{cloud computing} resulta conveniente, ya que son los grandes centros de datos los que deben cumplir el marco regulatorio. No obstante, la centralización aumenta el riesgo ante filtraciones o vulnerabilidades, ya que se tiene un tránsito constante de datos sensibles. Además, la cesión de derechos de uso puede ser un procedimiento extenso y costoso. \\

Al trasladar parte del procesamiento al borde, minimizamos el tránsito antes mencionado, ya que gran parte de los datos se procesa de forma local y la nube solo recibe resultados y/o metadatos \fcite{Cao2023}. En esta línea, los \textit{gateways edge} pueden implementar controles de acceso y cifrado de datos basados en la normativa correspondiente \fcite{Wang2024}. \\

Además, como el propio paciente puede acceder, rectificar y suprimir los datos en el nodo \textit{edge} local, así como oponerse a su cesión, se agiliza el cumplimiento de los derechos ARCO, regulados en la \citealp[Arts.~12–15, 18]{BOE}. \\

Con toda esta gestión de los datos en el paradigma de la computación en el borde, se desarrolla el aprendizaje federado, un \textit{framework} que permite entrenar modelos de gran calibre sin que los datos abandonen su origen. Se harán entrenamientos locales en los nodos \textit{edge} y la nube recibirá únicamente los parámetros resultantes. Agregando todos los resultados, se obtiene un modelo equivalente al que se hubiese obtenido haciendo un entrenamiento global con los datos centralizados \fcite{Rieke2020}. 

\section{Retos y líneas de trabajo futuras}

Si bien \textit{edge computing} resulta conveniente en muchos aspectos sobre \textit{cloud computing} y, en conjunto con este, da lugar a arquitecturas de gran utilidad en diversos ámbitos, este nuevo marco de computación aún tiene varios retos técnicos y logísticos que afrontar, así como consideraciones en materia de seguridad y privacidad \fcite{Rancea2024, Tlemcani2025, Veeramachaneni2025}. \\

Una preocupación habitual en este tipo de sistemas es la escalabilidad, ya que la infraestructura que soporta los dispositivos \textit{edge} se debe adaptar a un aumento en la cantidad de datos mientras mantiene un funcionamiento óptimo \fcite{Veeramachaneni2025}. Si mantenemos el foco en los propios dispositivos, el consumo de energía, así como la obsolescencia de los mismos, también se debe tener en cuenta en entornos remotos o de recursos limitados \fcite{Veeramachaneni2025}. Según se comenta en este artículo, en esta línea se están desarrollando soluciones de \textit{wake‑on‑demand} u \textit{offload selectivo} para alargar la vida útil de los dispositivos. \\

Un problema que surge por la propia naturaleza de la computación en el borde es debido a la heterogeneidad de dispositivos \textit{edge} y de recursos computacionales, lo que resulta en resultados inconsistentes entre distintos \textit{clusters} de IoMT/IoT \fcite{Rancea2024}. Para resolver esto, es necesario desarrollar sistemas capaces de gestionar diversos formatos de datos, protocolos de comunicación y estándares de seguridad. \\

Otro problema que resulta evidente es la capacidad limitada de los dispositivos IoMT/IoT, por lo que es necesario prestar especial atención a la optimización de los algoritmos que implementemos en ellos para el procesamiento de datos, ya sea en un mismo nodo con modelos ligeros que se ajusten a las limitaciones del dispositivo \textit{edge} \fcite{Veeramachaneni2025} o paralelizando la ejecución \fcite{Rancea2024}. \\

La seguridad y privacidad, que parecía mostrar grandes ventajas en \textit{edge computing}, también resulta un motivo de preocupación. Si bien no existe el riesgo de una gran exposición de datos, la baja seguridad en los dispositivos \textit{edge} los hace propensos a sufrir ciberataques y genera vulnerabilidades \fcite{Tlemcani2025, Veeramachaneni2025}. \\

En los últimos años se ha estado explorando la integración de \textit{blockchain} y \textit{edge computing} para mejorar la seguridad y privacidad de los datos clínicos de pacientes mediante una encriptación más robusta \fcite{Christo2021} e incluso reducir la dependencia de sistemas descentralizados \fcite{Nguyen2021}. Una revisión más exhaustiva de las innovaciones en salud y privacidad se puede encontrar en \fcite{Tlemcani2025}. \\

Otra tendencia la podemos ver en el desarrollo de modelos de IA distribuidos y federados, lo que permitirá entrenar modelos con mayores capacidades en los dispositivos \textit{edge} sin compartir los resultados \fcite{Rancea2024}. 

\section{Conclusión}

A lo largo de este trabajo vimos las características de \textit{edge computing} y cómo, en combinación con \textit{cloud computing}, dan lugar a una arquitectura de gran utilidad. Además, lo contextualizamos en el ámbito sanitario, junto con el IoMT. Tratamos en profundidad los beneficios que trae este marco de trabajo y computación de cara a la monitorización de pacientes crónicos, incluso en zonas rurales o de baja latencia o conexión irregular, debido a las capacidades de computación que tienen los dispositivos \textit{edge}. \\

Al tratar con datos clínicos, la privacidad y gobernanza de los mismos es un tema que debíamos tratar de forma explícita. Comentamos las legislaciones generales que los regulan y cómo \textit{edge computing} presenta tanto ventajas como desventajas respecto a la seguridad de estos datos. \\

Tras lo expuesto en la sección de retos y líneas de trabajo futuras, podemos concluir que, si bien es una tecnología con mucho que ofrecer al sector sanitario (de hecho, como hemos comentado, ya ha mostrado efectos y aplicaciones beneficiosas inmediatas), aún le queda mucho camino por recorrer. El futuro de la computación en el borde recae en su integración con \textit{blockchain} para mejorar la privacidad, con métodos de ahorro de energía, aprendizaje federado para el entrenamiento de modelos, o creación de sistemas robustos para tratar con la heterogeneidad de dispositivos IoMT.


\printbibliography[heading=bibintoc, title={Referencias}]

\end{document}



