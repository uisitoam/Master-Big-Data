\documentclass[11pt]{opticajnl}
\journal{opticajournal} % use for journal or Optica Open submissions

\usepackage[utf8]{inputenc}
\usepackage[T1]{fontenc} 
\usepackage{etoolbox}
\usepackage{comment}
\makeatletter

\setboolean{shortarticle}{true}

\usepackage{lineno}

%\linenumbers % Turn off line numbering for Optica Open preprint submissions.

\title{Redes 5G}

\author[1,2,3]{Luis Ardévol Mesa}


\begin{abstract}
En este trabajo se hablará acerca de cómo el 5G me jode la vida.
\end{abstract}

\setboolean{displaycopyright}{false} % Do not include copyright or licensing information in submission.

\begin{document}

\maketitle

\section{Introducción}

El 5G me jode.

\section{Metodología}

Para probar la hipótesis, tumbé la electricidad en toda Europa durante $\sim$ 24 horas. 

\section{Resultados}

Fui más feliz.

\section{Conclusión}

El 5G me jode. 

\begin{comment}

\section{Introducción con enfoque narrativo}

Inicia con un caso real o hipotético para captar la atención. Formula una pregunta clave: ¿Cómo puede MEC transformar la atención médica en tiempo real? Explica brevemente por qué IoT y Big Data han revolucionado la salud, pero también han traído nuevos desafíos (latencia, sobrecarga de datos, privacidad).

\section{Problema Central y Contexto Tecnológico}

IoT en Salud: Monitores cardíacos, sensores de glucosa, dispositivos portátiles y su necesidad de procesamiento eficiente.
Big Data en Salud: Generación masiva de datos clínicos y el reto de analizarlos en tiempo real.
Limitaciones de la Computación en la Nube: Latencia, dependencia de la conectividad, riesgos de privacidad.
MEC como Solución: Introducción básica del concepto, explicando que descentraliza el procesamiento para reducir latencias y mejorar eficiencia.

\section{Ejemplo Aplicado / Caso Ilustrativo}

Plantea un caso donde MEC mejora la atención médica:
Ejemplo 1: Hospital que implementa MEC para procesar imágenes médicas en el edge, reduciendo tiempos de diagnóstico.
Ejemplo 2: Dispositivos wearables que detectan anomalías cardíacas y procesan los datos localmente antes de enviarlos a los médicos.
Explica el impacto en términos de velocidad, precisión y seguridad.

\section{Aporte Personal: Reflexión Crítica y Mejora}

Desafíos actuales de MEC en Healthcare:
Implementación costosa.
Integración con sistemas hospitalarios existentes.
Seguridad de los datos sensibles.
Propuestas de mejora:
Optimización del uso de recursos en los nodos edge.
Modelos híbridos de MEC + Cloud.
Aplicación de IA en el edge para análisis de datos en tiempo real.
¿Es realmente viable para todos los sistemas de salud? Reflexiona sobre su aplicabilidad en hospitales pequeños vs. grandes centros médicos.
\section{Conclusión con Impacto}
Conectar con la idea inicial (cierre narrativo).
Reflexión sobre el futuro de MEC en la salud:
¿Será un estándar en el futuro?
¿Podría reemplazar por completo la computación en la nube?
Última pregunta para el lector: ¿Estamos listos para esta transformación?
Notas Finales
- Mantén un tono analítico pero accesible.
- Usa ejemplos claros y evita tecnicismos innecesarios.
- Enfatiza tu punto de vista y posibles mejoras en MEC.

Si necesitas pulir algún punto o darle más estructura a una sección específica, dime y seguimos ajustándolo. ¡Dale caña a ese trabajo! 
\end{comment}
\end{document}
