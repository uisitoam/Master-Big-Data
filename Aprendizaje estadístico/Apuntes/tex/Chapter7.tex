\chapter{KNN}\label{Chapter7} 
% chktex-file 8
% chktex-file 12
% chktex-file 13
% chktex-file 44


\begin{table}
\centering
\begin{tabular}{ccc}
 & superficie (m2) & distancia (km) \\
superficie & 70 & 0.1 \\
distancia & 90 & 5.0
\end{tabular}
\end{table}

Si tenemos un piso de 85 m2 y 0.4 km de distancia, intuitivamente se parece más al primer piso que al segundo. Pero si calculamos la distancia euclidiana, obtenemos que la distancia al primer piso esta mucho mas lejos que el segundo. Sin embargo, si cambiamos la escala de km a m, esto no sucede. Por esto hay que estandarizar:

\begin{equation}
x \to \frac{x - \bar{x}}{\text{std}x} = x'
\end{equation}

Asi, la media de x' es 0 y la desviacion estandar 1. Para dar la salida hay que desentandarizar !!!

Numero de vecino es el unico hiperparametro. Hay que estandarizar de forma obligatoria, pocos vecinos, sobreaprendizaje, muchos vecinos, subaprendizaje.

Pcos datos de entrenamiento y alta dimensionalidad de entrada es el peor escenario en aprendizaje estadistico, no le sienta bien a ningun algoritmo.