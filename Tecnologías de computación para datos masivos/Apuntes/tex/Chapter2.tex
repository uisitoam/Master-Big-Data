\chapter{Introducción a Hadoop}\label{Chapter2} 
% chktex-file 8
% chktex-file 12
% chktex-file 13
% chktex-file 44

\begin{enumerate}
\item  Nada
\item Libreria que permite programar las tareas: MapReduce.
\item Esta seria la idea de ocmo se hacen las aplicaciones e nhadoop. Map Reduce se usa por encima del almacenamiento deistribuidos. Pig y Hive (version de SQL) son lenguajes de alto nivel que se traducen a tareas MapReduce. Hay otros tipos de aplicaciones como SParK que no usan MapReduce, sino se ejecutan directamente sobre YARN. La base de datos no relacional HBase se ejecuta directamente sobre HDFS y hay otros proyectos como Mahout para machine learning, Flume para ingestar datos, etc.
\item La parte de instalacion la haremos en practica. Se puede probar en tu ordenador local con un unico nodo o de un modo pseudodistribuido.
\item La unica dificultad de la instalacion es configurarlo. Los 4 mas importantes son esos que vemos
\item Este es de ejemplo
\item Veamos las dos partes que tiene. La primera HDFS. No trabaja tan bien con ficheros pequeños. Latencia - cuanto tardamos en empezar a leerlo, ancho de banda - lo que tardamos en leerlo. Las modificaciones siempre van al final. Cada reducer debe tener su propio fuchero ya que no pueden escribir dos a la vez en uno.
\item en el master se ejecuta un demonio que se llama namenode, que es el que manteiene la info. En los esclavos se ejecuta un demonio que se llama datanode. El namenode es el que sabe donde estan los bloques de los ficheros. El datanode es el que tiene los bloques de los ficheros, pero no tienen idea sobre los ficheros.
\item el backup cada cierto tiempo realiza un checkpoint por si en namenode falla. 
\item el RM en el master y el NM en los esclavos. El AM se introduce mas adelante ya que cuando aumentaba mucho el numero de tareas la carga del RM era demasiado alta. USa figura de 17 y salta a esa.
\item (vuelve aqui desde la 17) y sigue. El RM se divide en dos, el scheduler y el applications manager AsM. Le da un poco igual 
\item no mucho
\item no mucho, ya lo dijo todo en la 17. YARN es un poco más complicado
\item El RM inicializa una tarea y justo asigna un AM. La aplicacion app mstr lanza los contenedores pidiendo al RM donde puede lanzar X tareas map y esas tareas, mientras se ejecutan, van hablandole al app master, que vuelve a hablar al RM para pedir mas espacio. Por lo que el RM se encarga de inicializar tareas y conceder recursos. 
\item blabla
\item blabla
\item comenta con el blabla
\item el app master pide contenedores con x recursos. 
\item Hadoop streaming supone una perdida de rendimeinto muy muy grande
\end{enumerate}