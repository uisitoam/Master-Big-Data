\section{Bases de datos NoSQL: Grafos (Neo4j)}

Neo4J viene ya instalado en la máquina virtual que estamos usando, por lo que simplemente debemos arrancar el servicio con el comando \texttt{sudo systemctl start neo4j}. Trabajaremos desde la interfaz \textit{web} en \texttt{http://localhost:7474}. Esta nos da celdas para introducir consultas en Cypher, el lenguaje de Neo4J, y proporciona una visualización gráfica de los resultados (si lo permiten) y en forma de tablas. \\

\subsection{Creación de índices}

Lo primero es crear la estructura de nodos y enlaces adecuada para las consultas. Iremos describiéndola poco a poco, pero antes introduciremos los índices necesarios para que las consultas pensadas sean eficientes:

\begin{minted}[frame=single, fontsize=\footnotesize]{cypher}
CREATE INDEX IF NOT EXISTS FOR (j:Jugador) ON (j.id);
CREATE INDEX IF NOT EXISTS FOR (et:EdicionTorneo) ON (et.torneo, et.fecha);
CREATE INDEX IF NOT EXISTS FOR (p:Partido) ON (p.num_partido, p.fecha);
CREATE INDEX IF NOT EXISTS FOR (p:Pais) ON (p.codigo_iso2);
CREATE INDEX IF NOT EXISTS FOR (t:Torneo) ON (t.id);
\end{minted}

La indexación de las originales claves primarias en las tablas de relacionales resulta en consultas mucho más eficientes. Esto lo hacemos para \texttt{Jugador}, \texttt{EdicionTorneo}, \texttt{Pais} y \texttt{Torneo}. Para facilitar la búsqueda de partidos específicos, se indexan simplemente el número de partido y la fecha.

\subsection{Carga de datos con JBDC}

Al estar los datos almacenados dentro de un SGBD relacional, usaremos JDBC para acceder directamente al gestor y recuperar los datos. A continuación, vamos creando la base de datos de grafos a la vez que cargamos los datos de la base de datos relacional. Iremos describiendo poco a poco los nodos y relaciones que compondrán la nueva BD. Comenzamos creando nodos para los países, con sus correspondientes atributos de la tabla relacional. 

\begin{minted}[frame=single, fontsize=\footnotesize]{cypher}
WITH "jdbc:postgresql://localhost:5432/tenis?user=alumnogreibd&password=greibd2021" as url
CALL apoc.load.jdbc(url, "SELECT codigo_iso2, codigo_iso3, codigo_ioc, nombre FROM pais") YIELD row
CREATE (p:Pais {
codigo_iso2: row.codigo_iso2,
codigo_iso3: row.codigo_iso3,
codigo_ioc: row.codigo_ioc,
nombre: row.nombre
});
\end{minted}


Creamos nodos para los jugadores, con sus correspondientes atributos de la tabla relacional. Se crea una relación \texttt{REPRESENTA\_A} entre el nodo del país y el nodo del jugador, para indicar que el jugador representa a su país. Para ello, se usa \texttt{MATCH} para encontrar el nodo del país correspondiente al jugador, se crea el nodo del jugador y, posteriormente, se crea la relación (saliente desde el nodo jugador hacia el nodo país).

\begin{minted}[frame=single, fontsize=\footnotesize]{cypher}
WITH "jdbc:postgresql://localhost:5432/tenis?user=alumnogreibd&password=greibd2021" AS url
CALL apoc.load.jdbc(url, 
    "SELECT id, nombre, apellido, diestro, fecha_nacimiento, pais, altura FROM jugador") YIELD row
MATCH (pa:Pais {codigo_iso2: row.pais})
CREATE (j:Jugador {
    id: row.id,
    nombre: row.nombre,
    apellido: row.apellido,
    diestro: row.diestro,
    fecha_nacimiento: row.fecha_nacimiento,
    altura: row.altura
})
CREATE (j)-[:REPRESENTA_A]->(pa);
\end{minted}

Nodos para los torneos, con sus correspondientes atributos de la tabla relacional. Se crea una relación \texttt{SE\_CELEBRA\_EN} entre el nodo del país y el nodo del torneo, para indicar que el torneo se celebra en un país. Para ello, se usa \texttt{MATCH} para encontrar el nodo del país correspondiente al torneo, se crea el nodo del torneo y, posteriormente, se crea la relación (saliente desde el nodo torneo hacia el nodo país).

\begin{minted}[frame=single, fontsize=\footnotesize]{cypher}
// cargamos los torneos
WITH "jdbc:postgresql://localhost:5432/tenis?user=alumnogreibd&password=greibd2021" AS url
CALL apoc.load.jdbc(url, "SELECT id, nombre, pais FROM torneo") YIELD row
CREATE (t:Torneo {
    id: row.id,
    nombre: row.nombre
})
WITH t, row
OPTIONAL MATCH (pa:Pais {codigo_iso2: row.pais})
WITH t, pa
WHERE pa IS NOT NULL
CREATE (t)-[:SE_CELEBRA_EN]->(pa);
\end{minted}

Nodos para las ediciones de los torneos, con sus correspondientes atributos de la tabla relacional. Se crea una relación \texttt{EDICION\_DE} entre el nodo de la edición del torneo y el nodo del torneo, para indicar que la edición del torneo pertenece a un cierto torneo. Para ello, se usa \texttt{MATCH} para encontrar el nodo del torneo correspondiente a la edición del torneo, se crea el nodo de la edición del torneo y, posteriormente, se crea la relación (saliente desde el nodo edición del torneo hacia el nodo torneo).

\begin{minted}[frame=single, fontsize=\footnotesize]{cypher}
// cargamos las ediciones de los torneos
WITH "jdbc:postgresql://localhost:5432/tenis?user=alumnogreibd&password=greibd2021" AS url
CALL apoc.load.jdbc(url, "SELECT torneo, fecha, superficie, tamano, nivel FROM edicion_torneo") YIELD row
MATCH (t:Torneo {id: row.torneo})
CREATE (et:EdicionTorneo {
    fecha: row.fecha,
    superficie: row.superficie,
    tamano: row.tamano,
    nivel: row.nivel,
    torneo: row.torneo
})
CREATE (et)-[:EDICION_DE]->(t);
\end{minted}

A continuación, cargamos los partidos creando nodos con sus correspondientes atributos de la tabla relacional. Esta creación dio varios problemas debido al conflicto con el \textit{left join} que comentamos en la sección de agregados; con una carga estándar se omitían partidos, y por tanto la consulta 4 no devolvía los resultados esperados. Para solucionar esto, se ha optado por crear primero el nodo del partido y, posteriormente, se crean las relaciones con los jugadores, torneos y ganadores/perdedores. La creación de las mismas la haremos con un \texttt{OPTIONAL MATCH} para que se cree la relación si existe el nodo al que se quiere enlazar, pero en caso negativo, se incluya el resto de datos del partido igualmente. Además, se ha añadido un \texttt{LIMIT} para evitar problemas de rendimiento: la carga se hizo de 10000 en 10000, cambiando el \texttt{OFFSET} en cada carga. \\

\noindent Vamos un poco más en detalle con las relaciones:
\begin{itemize}
\item \texttt{SE\_JUEGA\_EN}: Relación desde el nodo del partido hacia el nodo del torneo, para indicar que el partido se juega en un cierto torneo. Esta relación se crea con un \texttt{OPTIONAL MATCH} para que se cree la relación si existe el nodo del torneo al que se quiere enlazar, pero en caso negativo, se incluya el resto de datos del partido igualmente.
\item \texttt{GANADO\_POR}: Relación desde el nodo del partido hacia el nodo del jugador, para indicar que el partido ha sido ganado por un cierto jugador. Esta relación se crea con un \texttt{OPTIONAL MATCH} para que se cree la relación si existe el nodo del jugador al que se quiere enlazar, pero en caso negativo, se incluya el resto de datos del partido igualmente. Además, esta relación contiene como atributos las estadísticas del ganador en ese partido. 
\item \texttt{PERDIDO\_POR}: Relación completamente análoga a la anterior, pero para el perdedor del partido.
\end{itemize}

\begin{minted}[frame=single, fontsize=\footnotesize]{cypher}
// cargamos los partidos
WITH "jdbc:postgresql://localhost:5432/tenis?user=alumnogreibd&password=greibd2021" AS url
CALL apoc.load.jdbc(url,
"SELECT * FROM partido p LIMIT 10000 OFFSET 0") YIELD row

CREATE (p:Partido {
num_partido: row.num_partido,
fecha: row.fecha,
num_sets: row.num_sets,
ronda: row.ronda,
desenlace: row.desenlace,
torneo_id: row.torneo 
})

// vinculamos con el torneo si existe
WITH p, row
OPTIONAL MATCH (t:Torneo {id: row.torneo})
WITH p, row, t
WHERE t IS NOT NULL
CREATE (p)-[:SE_JUEGA_EN]->(t)

// vinculamos con el ganador si existe
WITH p, row
OPTIONAL MATCH (ganador:Jugador {id: toInteger(row.ganador)})
WITH p, row, ganador
WHERE ganador IS NOT NULL
CREATE (p)-[:GANADO_POR {
num_aces: row.num_aces_ganador,
num_dob_faltas: row.num_dob_faltas_ganador,
num_ptos_servidos: row.num_ptos_servidos_ganador,
num_primeros_servicios: row.num_primeros_servicios_ganador,
num_primeros_servicios_ganados: row.num_primeros_servicios_ganados_ganador,
num_segundos_servicios_ganados: row.num_segundos_servicios_ganados_ganador,
num_juegos_servidos: row.num_juegos_servidos_ganador,
num_break_salvados: row.num_break_salvados_ganador,
num_break_afrontados: row.num_break_afrontados_ganador
}]->(ganador)

// vinculamos con el perdedor si existe
WITH p, row
OPTIONAL MATCH (perdedor:Jugador {id: toInteger(row.perdedor)})
WITH p, row, perdedor
WHERE perdedor IS NOT NULL
CREATE (p)-[:PERDIDO_POR {
num_aces: row.num_aces_perdedor,
num_dob_faltas: row.num_dob_faltas_perdedor,
num_ptos_servidos: row.num_ptos_servidos_perdedor,
num_primeros_servicios: row.num_primeros_servicios_perdedor,
num_primeros_servicios_ganados: row.num_primeros_servicios_ganados_perdedor,
num_segundos_servicios_ganados: row.num_segundos_servicios_ganados_perdedor,
num_juegos_servidos: row.num_juegos_servidos_perdedor,
num_break_salvados: row.num_break_salvados_perdedor,
num_break_afrontados: row.num_break_afrontados_perdedor
}]->(perdedor);
\end{minted}

Por último, cargamos los sets de los partidos con los atributos de la tabla relacional homónima. Se crea una relación \texttt{PERTENECE\_A} desde el nodo del set del partido hacia el nodo del partido, para indicar que el set pertenece a un cierto partido. Para ello, se usa \texttt{OPTIONAL MATCH} para encontrar el nodo del partido correspondiente al set del partido, y, si se encuentra, crea el nodo del set del partido y, posteriormente, se crea la relación. La carga de estos datos también la haremos de 10000 en 10000 para evitar problemas de rendimiento.

\begin{minted}[frame=single, fontsize=\footnotesize]{cypher}
// cargamos los sets de los partidos
WITH "jdbc:postgresql://localhost:5432/tenis?user=alumnogreibd&password=greibd2021" AS url
CALL apoc.load.jdbc(url,
"SELECT * FROM sets_partido LIMIT 10000 OFFSET 0") YIELD row
OPTIONAL MATCH (p:Partido {num_partido: row.num_partido, fecha: row.fecha})
WITH row, p
WHERE p IS NOT NULL
CREATE (sp:SetPartido {
num_set: row.num_set,
juegos_ganador: row.juegos_ganador,
juegos_perdedor: row.juegos_perdedor,
puntos_tiebreak_perdedor: row.puntos_tiebreak_perdedor
})
CREATE (sp)-[:PERTENECE_A]->(p);
\end{minted}


\subsection{Consultas}

Con la base de datos de grafos ya creada, podemos realizar las mismas consultas que llevamos realizando durante todo el documento. Para aligerar la cantidad de texto en las explicaciones posteriores, vamos a comentar algunos detalles de la sintaxis del código, comunes a todas las consultas. De este modo, en la descripción de cada consulta podremos dar un enfoque menos técnico (al conocer ya de antemano esos detalles) y más semántico. 
\begin{itemize}
\item Para indicar una relación entre dos nodos, usamos \texttt{-->}. Con el signo $->$ indicamos la dirección de la relación, es decir, con $->$ indicacom que la relación va desde el nodo de la izquierda hacia el de la derecha, y con $<-$ indicamos que la relación va desde el nodo de la derecha hacia el de la izquierda.
\item La relación entre ambos nodos la especificaremos entre corchetes, como por ejemplo \texttt{-[:RELACION]->}, y podemos agregar un alias en caso de que queramos referirnos a la relación, por ejemplo \texttt{-[r:RELACION]->}. En caso de ser una relación con atributos, podemos añadirlos entre llaves, como por ejemplo \texttt{-[:RELACION \{atributo1: valor1, atributo2: valor2\}]->}.
\item Los nodos los especificamos entre paréntesis, por ejemplo \texttt{()}. Si queremos hacer referencia a ello, podemos añadir un alias, por ejemplo \texttt{(n)}. En caso de querer referirnos a un tipo de nodo concreto, añadimos el tipo, por ejemplo \texttt{(n:TipoNodo)}.
\item La estructura básica de las consultas comenzará con un \texttt{MATCH} para buscar patrones en el grafo, seguirá con un \texttt{WHERE} para filtrar los resultados (como haciamos en SQL), y finalmente un \texttt{RETURN} para mostrar los resultados. En varias consultas queremos hacer cálculos sobre los resultados; para ello usamos \texttt{WITH}, que nos permite pasar los resultados de una cláusula a otra. Al igual que en SQL, podemos usar \texttt{ORDER BY} para ordenar los resultados.
\end{itemize}

Con todo esto, si queremos, por ejemplo, buscar patrones de partidos ganados por un jugador en cierto torneo, usaremos un patrón de búsqueda en el \texttt{MATCH} de tipo 
\begin{equation*}
    \texttt{(j:Jugador)<-[:GANADO\_POR]-(p:Partido)-[:SE\_JUEGA\_EN]->(t:Torneo)}
\end{equation*}

Usamos las relaciones ya creadas, un partido es ganado por un jugador (relación desde el partido hacia el jugador) y se juega en un torneo (relación desde el partido hacia el torneo). Especificamos un alias para cada nodo e indicamos el tipo de nodo entre los paréntesis. \\

Tras esta breve introducción de los detalles sintácticos más básicos de este lenguaje, mostramos las consultas en Cypher que hemos realizado en Neo4J para obtener los mismos resultados que en las consultas SQL. \\

\subsubsection{Muestra todos los ganadores del torneo ``Wimbledon'' (Nombre apellidos y año). Ordena el resultado por año.}

Para esta consulta, curiosamente, nos sirve el ejemplo que pusimos antes. Queremos buscar jugadores que hayan ganado el partido de la ronda final de un torneo concreto, Wimbledon. Para esto, partimos de nodos tipo \texttt{Partido} y buscamos los nodos de tipo \texttt{Jugador} y de tipo \texttt{Torneo} que se relacionen con ese partido mediante las relaciones \texttt{GANADO\_POR} y \texttt{SE\_JUEGA\_EN}, respectivamente. Al buscar este patrón, debemos filtrar o seleccionar solo los nodos de partidos que correspondan a rondas finales, y el nodo de torneo Wimbledon; esto lo hacemos con el \texttt{where}, como en SQL. \\

Con el \texttt{RETURN} mostramos el nombre y apellido de los jugadores que cumplen con el patrón, así como el año de la edición del torneo en el que ganaron. Para obtener el año, convertimos la fecha a string, con \texttt{toString()}, y nos quedamos con los 4 primeros caracteres de esa cadena de texto, que son el año; esto lo conseguimos con \texttt{substring}, fijando como caracter inicial el 0 y final el 3 (4 no incluido). Finalmente, ordenamos el resultado de forma ascendente con el \texttt{ORDER BY ano}. A continuación se muestra la consulta en Cypher y el resultado en la figura \ref{fig:q1_neo}.

\begin{minted}[frame=single, fontsize=\footnotesize]{cypher}
MATCH (j:Jugador)<-[gp:GANADO_POR]-(p:Partido)-[:SE_JUEGA_EN]->(t:Torneo)
WHERE t.nombre = 'Wimbledon'
    AND p.ronda = 'F'
RETURN j.nombre AS nombre, j.apellido AS apellido, substring(toString(p.fecha), 0, 4) AS ano
ORDER BY ano;
\end{minted}

\begin{figure}[H]
\centering
\includegraphics[width=0.35\textwidth]{fotos/q1_neo.png}
\caption{Modelo de grafos en Neo4J, consulta 1. Al no caber la visualización de la tabla completa en la pantalla, se recortó la imagen por la mitad y se compone en dos partes para mostrar la tabla completa.}
\label{fig:q1_neo}
\end{figure}

\newpage


\subsubsection{Muestra los años en los que Roger Federer ganó algún torneo de nivel Gran Slam (G) o Master 1000 (M). Para cada año, muestra el número de torneos y lista sus nombres (ordenados por la fecha de celebración). Ordena el resultado por el año}

Comenzamos buscando los nodos de tipo \texttt{Partido} que se relacionen con los nodos de tipo \texttt{Jugador} y \texttt{Torneo} mediante las relaciones \texttt{GANADO\_POR} y \texttt{SE\_JUEGA\_EN}, respectivamente. Además, estamos interesados en las distintas ediciones de un mismo torneo, por lo que buscamos los nodos de tipo \texttt{EdicionTorneo} que se relacionen con los nodos de tipo \texttt{Torneo} mediante la relación \texttt{EDICION\_DE}. Concretamente, filtraremos los nodos de partidos que correspondan a rondas finales, y los nodos de ediciones de torneos que tengan un nivel de `G' o `M'. Además, nos aseguramos de que la fecha de la edición del torneo sea la misma que la fecha del partido. \\

Hasta aquí tendríamos ya todo lo necesario para devolver los datos que se piden, pero necesitamos agrupar los resultados por año y ordenar, dentro de un mismo año, los torneos según su fecha de celebración. Para esto vamos fijando resultados intermedios:
\begin{itemize}
\item La agrupación por año se hace con el primer \texttt{WITH}, donde obtenemos el año a partir de la fecha de celebración del partido (como hicimos en la consulta anterior). Tras esto, \texttt{collect()} nos permite agrupar los torneos de un mismo año (al haber realizado el \texttt{WITH} antes). Para ordenarlos por fecha, guardamos tanto los nombres como las fechas de los partidos. 
\item El conjunto de torneos de cada año se almacena en una colección y, para operar con ella, debemos deshacerla en filas; esto lo conseguimos con \texttt{UNWIND}. Como paso intermedio, nos quedamos con el año, el nombre y la fecha de cada torneo, y ahora sí podemos hacer un \texttt{ORDER BY} sobre la fecha de celebración de cada torneo. Decidimos usar \texttt{UNWIND} ya que, en este caso, simplificaba la declaración de los campos a devolver en el \texttt{RETURN}, pero se podría haber prescindido de él. Veremos un ejemplo en la consulta 3.
\end{itemize}

Tras este ordenamiento usamos \texttt{RETURN} para mostrar el año, el número de torneos (como un conteo de los nombres distintos) y los nombres de los torneos ganados por Federer ese año, que se concatenan en una sola cadena con \texttt{reduce()}. \\

Para la concatenación usamos una función de reducción (sí, un poco avanzado para el nivel que estamos manejando en estas consultas, pero es la herramienta más adecuada que encontramos para sustituir al \texttt{STRING\_AGG} de SQL). Daremos una explicación detallada de su funcionamiento, ya que la usaremos en otras consultas. Por hacer una analogía, esta función sería como usar una función lambda en Python para condensar un bucle \texttt{for}: agrupamos los nombres de los torneos de un mismo año, ya ordenados, con \texttt{collect(nombres)}. Cogemos el primer nombre con \texttt{head()}, lo guardamos en una variable \texttt{s} y guardamos el resto de nombres en una lista \texttt{tail()}. Luego, concatenamos s con el siguiente elemento de x. Esto se repite de forma iterativa gracias al \texttt{reduce()} (que hace el papel del \texttt{for}) que rodea todo este bloque de código. A continuación se muestra la consulta en Cypher y el resultado en la figura \ref{fig:q2_neo}.

\begin{minted}[frame=single, fontsize=\footnotesize]{cypher}
MATCH (j:Jugador)<-[:GANADO_POR]-(p:Partido)-[:SE_JUEGA_EN]->(t:Torneo)
MATCH (et:EdicionTorneo)-[:EDICION_DE]->(t)
WHERE j.nombre = 'Roger'
    AND j.apellido = 'Federer'
    AND p.ronda = 'F'
    AND et.nivel IN ['G', 'M']
    AND et.fecha = p.fecha
    AND et.torneo = t.id
WITH substring(toString(p.fecha), 0, 4) AS ano, 
     collect({nombre: t.nombre, fecha: p.fecha}) AS torneos
UNWIND torneos AS torneo
WITH ano, torneo.nombre AS nombre, torneo.fecha AS fecha
ORDER BY fecha
RETURN ano, count(DISTINCT nombre) AS num_torneos, 
    reduce(s = head(collect(nombre)), x IN tail(collect(nombre)) | s + ', ' + x) AS torneos
ORDER BY ano;
\end{minted}

\begin{figure}[H]
\centering
\includegraphics[width=0.45\textwidth]{fotos/q2_neo.png}
\caption{Modelo de grafos en Neo4J, consulta 2.}
\label{fig:q2_neo}
\end{figure}



\subsubsection{Muestra los partidos de semifinales (ronda='SF') y final (ronda = 'F') del torneo de "Roland Garros" del 2018. Para cada partido muestra la ronda, el tipo de desenlace, el nombre y apellidos del ganador y el nombre y apellidos del perdedor y el resultado con el número de juegos del ganador y del perdedor en cada set, y opcionalmente en paréntesis el número de juegos del perdedor en el tie break}

Para esta consulta, comenzamos con los nodos de tipo \texttt{Partido}, desde los cuales buscamos relaciones de tipo \texttt{GANADO\_POR} y \texttt{PERDIDO\_POR} con los nodos de tipo \texttt{Jugador}, ya que queremos la información de ambos participantes. Además, desde estos nodos de partidos buscamos nodos de tipo \texttt{Torneo} mediante la relación \texttt{SE\_JUEGA\_EN}, y nodos de tipo \texttt{SetPartido} mediante la relación \texttt{PERTENECE\_A} (dirigida desde los \texttt{SetPartido} hacia los \texttt{Partido}). Este es el patrón inicial de búsqueda en nuestro grafo. Ahora debemos filtrar los resultados: 
\begin{itemize}
\item Del tipo \texttt{Torneo} queremos el nodo correspondiente a Roland Garros.
\item De los nodos de tipos \texttt{Partido} queremos aquellos que correspondan a las rondas de semifinales y final, y que sean del año 2018.
\end{itemize}

Con esto tenemos los nodos que queremos, así que con una claúsula \texttt{WITH} nos quedamos con ellos a partir de sus respectivos alias, los ordenamos por el número de set y, en otro \texttt{WITH}, concatenamos los resultados de los juegos de cada set en una sola cadena con el formato adecuado e incluyendo la puntuación del \textit{tie break} en caso de existir. Finalmente, con el \texttt{RETURN} mostramos la ronda y el desenlace, el nombre y apellidos de ambos jugadores (ganador y perdedor) que, al ser una concatención de dos valores, podemos hacer de forma más cómoda que con el \texttt{reduce()} de la consulta anterior. No obstante, volvemos a recurrir a la función de reducción para concatenar los resultados de los sets de cada partido. \\

Para esta consulta podríamos haber recurrido de nuevo a \texttt{UNWIND}, pero esta vez optamos por una forma más directa. Esto nos permite explorar varias formas de realizar una misma tarea, la agrupación ordenada. A continuación se muestra la consulta en Cypher y el resultado en la figura \ref{fig:q3_neo}.

\newpage

\begin{minted}[frame=single, fontsize=\footnotesize]{cypher}
MATCH (jg:Jugador)<-[:GANADO_POR]-(p:Partido)-[:PERDIDO_POR]-(jp:Jugador),
(p)-[:SE_JUEGA_EN]->(t:Torneo),
(sp:SetPartido)-[:PERTENECE_A]->(p)
WHERE t.nombre = 'Roland Garros'
    AND p.ronda IN ['SF', 'F']
    AND substring(toString(p.fecha), 0, 4) = '2018'
WITH p, jg, jp, sp
ORDER BY sp.num_set
WITH p, jg, jp,
    collect(sp.juegos_ganador + '-' + sp.juegos_perdedor +
    CASE sp.puntos_tiebreak_perdedor
    WHEN null THEN ''
    ELSE '(' + toString(sp.puntos_tiebreak_perdedor) + ')'
    END) as sets
RETURN p.ronda as ronda, p.desenlace as desenlace,
    jg.nombre + ' ' + jg.apellido as ganador,
    jp.nombre + ' ' + jp.apellido as perdedor,
    reduce(s = head(sets), x IN tail(sets) | s + ', ' + x) as resultado
ORDER BY p.fecha;
\end{minted}

\begin{figure}[H]
\centering
\includegraphics[width=0.65\textwidth]{fotos/q3_neo.png}
\caption{Modelo de grafos en Neo4J, consulta 3.}
\label{fig:q3_neo}
\end{figure}





\subsubsection{Muestra la lista de jugadores españoles (ES) que ganaron algún torneo de nivel Gran Slam (G). Para cada jugador muestra los siguientes datos resumen de todos sus partidos: número de partidos jugados, porcentaje de victorias, porcentaje de aces, porcentaje de dobles faltas, porcentaje de servicios ganados, porcentaje de restos ganados, porcentaje de break points salvados (de los sufridos en contra), porcentaje de break points ganados (de los provocados a favor)}

Comenzamos buscando jugadores españoles ganadores de un Grand Slam. Para ello, partimos de los nodos de tipo \texttt{Jugador} y buscamos los nodos de tipo \texttt{Pais} que se relacionen con ellos mediante la relación \texttt{REPRESENTA\_A}. A partir de estos nodos de jugadores, buscamos los nodos de tipo \texttt{Partido} que se relacionen con ellos mediante las relaciones \texttt{GANADO\_POR} (relación dirigida desde \texttt{Partido} hacia \texttt{Jugador}). Además, desde estos nodos de partidos buscamos los nodos de tipo \texttt{Torneo} mediante la relación \texttt{SE\_JUEGA\_EN}, y los nodos de tipo \texttt{EdicionTorneo} que se relacionen con los nodos de \texttt{Torneo} mediante la relación \texttt{EDICION\_DE}. Tras marcar el patrón de búsqueda, simplemente filtramos los nodos de partidos correspondientes a rondas finales y de ediciones de torneos de nivel `G', asegurando la coincidencia de las fechas de los partidos y las ediciones de los torneos. Con esto tenemos los jugadores que queriamos, que guardamos con un \texttt{WITH} para usar ahora en el cálculo de estadísticas. \\

Para calcular las estadísticas de estos jugadores, comenzamos por buscar el nodo de tipo \texttt{Jugador} correspondiente a cada jugador que hemos encontrado mediante un patrón de búsqueda sencillo donde especificamos el tipo de nodo en el \texttt{MATCH} y filtramos por el \texttt{id} del jugador. Luego buscamos los nodos de partido que se relacionen con el nodo del jugador que estemos considerando mediante la relación \texttt{GANADO\_POR} o \texttt{PERDIDO\_POR} (ya que queremos todos sus partidos). Necesitamos también las estadísticas de sus rivales, para calcular el número de restos de nuestro jugador, por lo que encajamos un patrón de búsqueda para los nodos de nuestros partidos que se relacionen con los nodos de tipo \texttt{Jugador} mediante las relaciones \texttt{PERDIDO\_POR} o \texttt{GANADO\_POR}, y los denotaremos por un alias distinto, para mantener las estadísticas de nuestro jugador y del rival separadas. Esto lo hacemos un \texttt{OPTIONAL MATCH} ya que puede darse el caso de que el rival no esté en la base de datos y sea NULL. \\

Sacaremos primero como resultado intermedio los valores que nos interesan para las estadísticas, y luego calcularemos las estadísticas que se piden. En el primer \texttt{WITH} nos quedamos con el nombre y apellidos de nuestro jugador, ya concatenado de antes, los nodos de partidos y jugador, la relación de nuestro jugador con el partido y el tipo de esta relación (ya que nos permitirá obtener el porcentaje de victorias), y la relación del rival con el partido. También nos quedamos con algunos valores estadísticos: 
\begin{itemize}
\item Con el tipo de relación podemos obtener una lista de 1 para los partidos ganados y 0 para los perdidos por nuestro jugador, como haciamos en SQL.
\item Nos quedamos con \texttt{num\_aces}, \texttt{num\_ptos\_servidos}, \texttt{num\_dob\_faltas}, \texttt{num\_primeros\_servicios\_ganados} y \texttt{num\_segundos\_servicios\_ganados} (concretamente la suma de estos dos últimos). Todo esto lo obtenemos fácilmente de la relación de nuestro jugador con cada partido, que denotamos con el alias \texttt{r}.
\item Para el resto de valores que necesitamos (que no vamos a explicar, ya que el cálculo de las estadísticas es el mismo que en SQL), usamos un \texttt{CASE} sobre el tipo de la relación de nuestro jugador con el partido, para saber si accedemos a los valores disponibles en la relación de nuestro rival con el partido de tipo \texttt{PERDIDO\_POR} o de tipo \texttt{GANADO\_POR}.
\end{itemize}

Con todo esto en el primer \texttt{WITH}, calculamos las estadísticas en el segundo \texttt{WITH}, donde nos quedamos con el nombre y apellido, el conteo de partidos (a partir del número de nodos de tipo \texttt{Partido} que tenga asociados nuestro jugador), y los porcentajes, que calculamos usando \texttt{sum()} sobre los valores que hemos obtenido en el primer \texttt{WITH} y dividiendo por la cantidad adecuada (de nuevo, esto son detalles de cálculo ya explicados, no de la propia consulta en Cypher). \\

Finalmente, con el \texttt{RETURN} mostramos el nombre y apellidos del jugador, el número de partidos jugados y las estadísticas, redondeadas a un decimal. A continuación se muestra la consulta en Cypher y el resultado en la figura \ref{fig:q4_neo}.

\begin{minted}[frame=single, fontsize=\footnotesize]{cypher}
// buscamos jugadores españoles que han ganado finales de Grand Slam (para eliminar duplicados, 
sino lo haciamos directamente con las estadísticas)
MATCH (j:Jugador)-[:REPRESENTA_A]->(p:Pais {codigo_iso2: 'ES'})
MATCH (j)<-[:GANADO_POR]-(partido:Partido)-[:SE_JUEGA_EN]->(t:Torneo)
MATCH (et:EdicionTorneo)-[:EDICION_DE]->(t)
WHERE partido.ronda = 'F'
    AND et.nivel = 'G'
    AND et.fecha = partido.fecha
    AND et.torneo = t.id
WITH DISTINCT j.id AS id_jugador, j.nombre + ' ' + j.apellido AS jugador

// calculamos las estadísticas de estos jugadores
MATCH (j:Jugador)
WHERE j.id = id_jugador
MATCH (p:Partido)
MATCH (j)<-[r:GANADO_POR|PERDIDO_POR]-(p)
// necesitamos las estadisticas de sus rivales
OPTIONAL MATCH (p)-[rp:PERDIDO_POR]->(rival_j:Jugador)
OPTIONAL MATCH (p)-[rg:GANADO_POR]->(rival_g:Jugador)
WITH jugador, p, j, r, type(r) AS tipo, rp, rg,
    CASE type(r) WHEN 'GANADO_POR' THEN 1 ELSE 0 END AS es_ganador,
    r.num_aces AS aces,
    r.num_ptos_servidos AS ptos_servidos,
    r.num_dob_faltas AS dobles_faltas,
    r.num_primeros_servicios_ganados + r.num_segundos_servicios_ganados AS servicios_ganados,
    // restos ganados
    CASE type(r)
    WHEN 'GANADO_POR' THEN rp.num_ptos_servidos - rp.num_primeros_servicios_ganados - 
    rp.num_segundos_servicios_ganados
    ELSE rg.num_ptos_servidos - rg.num_primeros_servicios_ganados - 
    rg.num_segundos_servicios_ganados END AS restos_ganados,
    CASE type(r)
    WHEN 'GANADO_POR' THEN rp.num_ptos_servidos
    ELSE rg.num_ptos_servidos END AS ptos_servidos_rival,
    r.num_break_salvados AS breaks_salvados,
    r.num_break_afrontados AS breaks_afrontados,
    CASE type(r)
    WHEN 'GANADO_POR' THEN rp.num_break_afrontados - rp.num_break_salvados
    ELSE rg.num_break_afrontados - rg.num_break_salvados END AS breaks_ganados,
    CASE type(r)
    WHEN 'GANADO_POR' THEN rp.num_break_afrontados
    ELSE rg.num_break_afrontados END AS breaks_rival

WITH jugador, count(p) AS partidos,
    100.0 * sum(es_ganador) / count(p) AS pcje_victorias,
    CASE WHEN sum(ptos_servidos) = 0 THEN 0
    ELSE 100.0 * sum(aces) / sum(ptos_servidos) END AS pcje_aces,
    CASE WHEN sum(ptos_servidos) = 0 THEN 0
    ELSE 100.0 * sum(dobles_faltas) / sum(ptos_servidos) END AS pcje_dobles_faltas,
    CASE WHEN sum(ptos_servidos) = 0 THEN 0
    ELSE 100.0 * sum(servicios_ganados) / sum(ptos_servidos) END AS pcje_servicios_ganados,
    CASE WHEN sum(ptos_servidos_rival) = 0 THEN 0
    ELSE 100.0 * sum(restos_ganados) / sum(ptos_servidos_rival) END AS pcje_restos_ganados,
    CASE WHEN sum(breaks_afrontados) = 0 THEN 0
    ELSE 100.0 * sum(breaks_salvados) / sum(breaks_afrontados) END AS pcje_breaks_salvados,
    CASE WHEN sum(breaks_rival) = 0 THEN 0
    ELSE 100.0 * sum(breaks_ganados) / sum(breaks_rival) END AS pcje_breaks_ganados
RETURN jugador, partidos,
    round(pcje_victorias, 1) AS pcje_victorias,
    round(pcje_aces, 1) AS pcje_aces,
    round(pcje_dobles_faltas, 1) AS pcje_dobles_faltas,
    round(pcje_servicios_ganados, 1) AS pcje_servicios_ganados,
    round(pcje_restos_ganados, 1) AS pcje_restos_ganados,
    round(pcje_breaks_salvados, 1) AS pcje_breaks_salvados,
    round(pcje_breaks_ganados, 1) AS pcje_breaks_ganados
ORDER BY jugador;
\end{minted}

\begin{figure}[H]
\centering
\includegraphics[width=\textwidth]{fotos/q4_neo.png}
\caption{Modelo de grafos en Neo4J, consulta 4.}
\label{fig:q4_neo}
\end{figure}





\subsubsection{Lista los jugadores que fueron derrotados (en algún partido del 2018) por el rival de Rafael Nadal de la primera ronda (R128) de Roland Garros de 2018}

Comenzamos la consulta buscando el nodo de tipo \texttt{Jugador} correspondiente a Rafael Nadal. Tomamos el resto de nodos de tipo \texttt{Jugador} y buscamos un patrón de nodos de tipo \texttt{Partido} que se relacionen con los nodos de tipo \texttt{Torneo} mediante la relación \texttt{SE\_JUEGA\_EN}. Filtramos los nodos de partidos correspondientes a la ronda R128 y al año 2018. Además, queremos quedarnos con los nodos de tipo \texttt{Partido} que se relacionen con el nodo de Rafael Nadal mediante las relaciones \texttt{GANADO\_POR} o \texttt{PERDIDO\_POR}, y con otro nodo de tipo \texttt{Jugador} mediante la relación opuesta. Nos quedamos con el nodo de este jugador rival. \\

A continuación, buscamos un patrón de nodos de tipo \texttt{Partido} que se relacionen con el nodo del rival mediante las relaciones \texttt{GANADO\_POR}, y con nodos de tipo \texttt{Jugador} mediante la relación \texttt{PERDIDO\_POR}. Filtramos los nodos de partidos correspondientes al año 2018. Nos quedamos con el nodo de este jugador perdedor y buscamos el nodo de tipo \texttt{Pais} con el que se relacione el perdedor mediante la relación \texttt{REPRESENTA\_A}. \\

Finalmente, con el \texttt{RETURN} mostramos el nombre y apellidos concatenados de los jugadores que encajen con el patrón de búsqueda y los filtros anteriores, junto con el código ISO2 del país al que representa. A continuación se muestra la consulta en Cypher y el resultado en la figura \ref{fig:q5_neo}.

\begin{minted}[frame=single, fontsize=\footnotesize]{cypher}
// encontramos al rival de Nadal en Roland Garros 2018 R128
MATCH (nadal:Jugador {nombre: 'Rafael', apellido: 'Nadal'})
MATCH (rival:Jugador)
MATCH (p:Partido)-[:SE_JUEGA_EN]->(t:Torneo {nombre: 'Roland Garros'})
WHERE p.ronda = 'R128'
AND substring(toString(p.fecha), 0, 4) = '2018'
AND ((nadal)<-[:GANADO_POR]-(p)-[:PERDIDO_POR]->(rival) OR
(nadal)<-[:PERDIDO_POR]-(p)-[:GANADO_POR]->(rival))


// buscamos los partidos donde este rival perdió en 2018
WITH rival
MATCH (rival)<-[:GANADO_POR]-(derrotas:Partido)-[:PERDIDO_POR]->(perdedor:Jugador)
WHERE substring(toString(derrotas.fecha), 0, 4) = '2018'
MATCH (perdedor)-[:REPRESENTA_A]->(pais:Pais)
RETURN DISTINCT perdedor.nombre + ' ' + perdedor.apellido as jugador, pais.codigo_iso2 as pais
ORDER BY jugador;
\end{minted}

\begin{figure}[H]
\centering
\includegraphics[width=0.25\textwidth]{fotos/q5_neo.png}
\caption{Modelo de grafos en Neo4J, consulta 5.}
\label{fig:q5_neo}
\end{figure}