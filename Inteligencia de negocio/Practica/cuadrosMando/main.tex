\documentclass[11pt]{opticajnl}
\journal{opticajournal} % use for journal or Optica Open submissions

% See template introduction for guidance on setting shortarticle option
\setboolean{shortarticle}{true}
% true = letter/tutorial
% false = research/review article

% ONLY applicable for journal submission shortarticle types:
% When \setboolean{shortarticle}{true}
% then \setboolean{memo}{true} will print "Memorandum" on title page header
% Otherwise header will remain as "Letter"
% \setboolean{memo}{true}
\usepackage{lineno}
\usepackage{listings}
\usepackage{xcolor}  % Para colorear el código

% Definir colores para la sintaxis de Python
\definecolor{pystring}{RGB}{186,33,33}      % Strings
\definecolor{pycomment}{RGB}{107,107,107}   % Comentarios
\definecolor{pykeyword}{RGB}{0,119,170}     % Palabras clave
\definecolor{pybuiltin}{RGB}{163,21,21}     % Funciones builtin
\definecolor{pybackground}{RGB}{250,250,250} % Fondo

% Definir el estilo para código Python
\lstdefinestyle{Python}{
    language=Python,
    backgroundcolor=\color{pybackground},
    basicstyle=\ttfamily\small,
    breakatwhitespace=false,
    breaklines=true,
    captionpos=b,
    keepspaces=true,
    numbers=left,
    numbersep=5pt,
    showspaces=false,
    showstringspaces=false,
    showtabs=false,
    tabsize=4,
    frame=single,
    commentstyle=\color{pycomment},
    keywordstyle=\color{pykeyword},
    stringstyle=\color{pystring},
    identifierstyle=\color{black},
    numberstyle=\tiny\color{gray},
    emphstyle=\color{pybuiltin},
    morekeywords={import,from,as,def,class,return,yield,for,while,if,else,elif,
                  try,except,finally,with,lambda,assert,pass,break,continue,
                  raise,global,nonlocal,True,False,None,and,or,not,is,in},
    emph={pandas,numpy,matplotlib,seaborn,sklearn,tensorflow,torch,
          plt,pd,np,sns,print,len,range,enumerate,zip,dict,list,tuple,set,
          min,max,sum,sorted,map,filter},
}

\lstdefinestyle{sql}{
  language=SQL,
  showspaces=false, 
  basicstyle=\ttfamily,
  numbers=left,
  numberstyle=\tiny,
  backgroundcolor=\color{gray!10},
  keywordstyle=\color{blue}\bfseries,
  commentstyle=\color{green!70}\ttfamily,
  stringstyle=\color{red!70}\ttfamily,
  breaklines=true,
  morekeywords={CREATE, DROP, TABLE, SELECT, INSERT, UPDATE, DELETE, FROM, WHERE, JOIN, ON, AS}, % Puedes añadir más palabras clave específicas de SQL.
  literate={``}{``}1 {''}{''}1 {“}{``}1 {”}{''}1 {‘}{`}1 {’}{'}1
}

\lstdefinestyle{terminal}{
  backgroundcolor=\color{white},   % Fondo blanco
  basicstyle=\color{black}\ttfamily, % Texto negro en fuente monoespaciada
  keywordstyle=\color{blue}\bfseries, % Palabras clave en azul y negrita
  commentstyle=\color{green!70}\ttfamily, % Comentarios en verde
  stringstyle=\color{red}\ttfamily, % Cadenas en rojo
  morekeywords={sudo, apt-get, install, cd, ls, mkdir, rm, rmdir, cp, mv, echo, cat, nano, vim, grep, find, chmod, chown, systemctl, service, update, upgrade, reboot, shutdown, exit}, % Comandos comunes de terminal
  breaklines=true, % Permitir saltos de línea
  frame=single, % Marco alrededor del código
  framerule=0.5pt, % Grosor del marco
  rulecolor=\color{gray}, % Color del marco
  xleftmargin=0.05\textwidth, % Margen izquierdo
  xrightmargin=0.05\textwidth, % Margen derecho
  aboveskip=1em, % Espacio antes del bloque de código
  belowskip=1em % Espacio después del bloque de código
}
\definecolor{rstring}{RGB}{186,33,33}     % Strings
\definecolor{rcomment}{RGB}{0,128,0}      % Comentarios
\definecolor{rfunction}{RGB}{0,0,255}      % Funciones
\definecolor{rkeyword}{RGB}{145,0,145}    % Palabras clave
\definecolor{rbackground}{RGB}{248,248,248} % Fondo

\lstdefinestyle{R}{
    language=R,
    backgroundcolor=\color{rbackground},
    basicstyle=\ttfamily\small,
    breakatwhitespace=false,
    breaklines=true,
    captionpos=b,
    keepspaces=true,
    numbers=left,
    numbersep=5pt,
    showspaces=false,
    showstringspaces=false,
    showtabs=false,
    tabsize=2,
    frame=single,
    commentstyle=\color{rcomment},
    keywordstyle=\color{rkeyword},
    stringstyle=\color{rstring},
    identifierstyle=\color{black},
    numberstyle=\tiny\color{gray},
    morekeywords={library, data.frame, read.csv, ggplot, aes, geom_bar, theme_minimal, 
                  scale_fill_manual, gather, group_by, summarise, arrange, filter},
}


%\linenumbers % Turn off line numbering for Optica Open preprint submissions.

\title{Cuadros de mando}

\author[1,2,3]{Luis Ardévol Mesa}


\begin{abstract}
A
\end{abstract}

\setboolean{displaycopyright}{false} % Do not include copyright or licensing information in submission.

\begin{document}

\maketitle
\section{Desarrollo de la Sección 6: Ranking de Directores y Productores según ROI Promedio (Con Dos Gráficos)}

\textbf{Descripción:}

Esta sección se enfoca en analizar el \textbf{Retorno de Inversión (ROI) promedio} de los proyectos realizados por combinaciones de \textbf{directores} y \textbf{productores}. Al identificar las combinaciones con mayor ROI, la productora puede tomar decisiones informadas sobre futuras colaboraciones y asignar recursos de manera óptima.

\textbf{Objetivos:}

\begin{itemize}
    \item Identificar dúos creativos rentables.
    \item Analizar tendencias en el desempeño financiero de colaboraciones específicas.
    \item Visualizar los datos desde diferentes perspectivas para obtener \textit{insights} más profundos.
\end{itemize}

\subsection{1. Consulta MDX para Calcular el ROI Promedio}

Utilizaremos una consulta MDX que calcule el ROI promedio por combinación de director y productor.

\begin{lstlisting}[style=terminal]
WITH 
-- Calculo del Beneficio
MEMBER [Measures].[Beneficio] AS 
  [Measures].[Ingresos] - [Measures].[Coste]

-- Calculo del ROI individual
MEMBER [Measures].[ROI] AS 
  IIF([Measures].[Coste] = 0, NULL, [Measures].[Beneficio] / [Measures].[Coste])

-- Calculo del ROI Promedio
MEMBER [Measures].[ROI Promedio] AS 
  AVG(
    [Tiempo].[Ano].Members,
    [Measures].[ROI]
  )

SELECT 
  {[Measures].[ROI Promedio]} ON COLUMNS,
  NONEMPTY(
    CROSSJOIN(
      [Director].[Nombre].[Nombre].Members,
      [Productor].[Nombre].[Nombre].Members
    )
  ) ON ROWS
FROM [Finanzas]
ORDER BY [Measures].[ROI Promedio] DESC
\end{lstlisting}

\subsection{2. Implementación en el Cuadro de Mando con Dos Gráficos}

Para enriquecer el análisis, implementaremos dos gráficos diferentes:

\begin{enumerate}
    \item \textbf{Gráfico de Barras Horizontales}: Muestra las combinaciones de directores y productores ordenadas por su ROI promedio.
    \item \textbf{Gráfico de Dispersión}: Visualiza la relación entre el número de proyectos realizados y el ROI promedio de cada dúo, identificando patrones y anomalías.
\end{enumerate}

\subsubsection{Gráfico 1: Ranking de Combinaciones por ROI Promedio}

\textbf{Tipo de Gráfico:} Gráfico de barras horizontales.

\textbf{Descripción:}

\begin{itemize}
    \item \textbf{Eje X (Horizontal):} Valor del ROI Promedio.
    \item \textbf{Eje Y (Vertical):} Combinaciones de Director y Productor (por ejemplo, ``Christopher Nolan - Emma Thomas'').
    \item \textbf{Ordenación:} De mayor a menor ROI Promedio.
\end{itemize}

\textbf{Implementación:}

\begin{itemize}
    \item \textbf{Visualización:} Muestra los dúos creativos más rentables.
    \item \textbf{Interactividad:}
    \begin{itemize}
        \item \textbf{Filtros:} Permitir filtrar por año, género, país u otros criterios relevantes.
        \item \textbf{Detalles al Clic:} Al seleccionar una barra, mostrar información adicional como:
        \begin{itemize}
            \item Lista de películas realizadas por la combinación.
            \item Ingresos y costes totales.
            \item Tendencia del ROI a lo largo del tiempo para ese dúo.
        \end{itemize}
    \end{itemize}
\end{itemize}

\textbf{Beneficios:}

\begin{itemize}
    \item Identificación rápida de las combinaciones más rentables.
    \item Facilita la comparación entre diferentes dúos.
\end{itemize}

\subsubsection{Gráfico 2: Relación entre Número de Proyectos y ROI Promedio}

\textbf{Tipo de Gráfico:} Gráfico de dispersión (scatter plot).

\textbf{Descripción:}

\begin{itemize}
    \item \textbf{Eje X (Horizontal):} Número de proyectos realizados por la combinación de director y productor.
    \item \textbf{Eje Y (Vertical):} ROI Promedio de la combinación.
    \item \textbf{Puntos Representados:} Cada punto representa una combinación de director y productor.
\end{itemize}

\textbf{Implementación:}

\begin{itemize}
    \item \textbf{Visualización:} Permite identificar si las combinaciones que han colaborado en más proyectos tienden a tener un ROI más alto o si existe algún patrón interesante.
    \item \textbf{Interactividad:}
    \begin{itemize}
        \item \textbf{Hover Tooltip:} Al pasar el cursor sobre un punto, mostrar:
        \begin{itemize}
            \item Nombres del director y productor.
            \item Número de proyectos.
            \item ROI Promedio.
        \end{itemize}
        \item \textbf{Filtros:} Posibilidad de filtrar por rangos de número de proyectos o ROI.
        \item \textbf{Zonas Destacadas:} Utilizar colores o áreas sombreadas para resaltar cuadrantes específicos (por ejemplo, alta cantidad de proyectos y alto ROI).
    \end{itemize}
\end{itemize}

\textbf{Beneficios:}

\begin{itemize}
    \item Identifica correlaciones entre experiencia conjunta (número de proyectos) y rentabilidad.
    \item Destaca combinaciones con alto ROI pero pocos proyectos, que pueden ser oportunidades emergentes.
\end{itemize}

\subsection{3. Pasos para Crear las Visualizaciones}

\textbf{Para ambos gráficos:}

\begin{enumerate}
    \item \textbf{Configurar la Fuente de Datos:}
    \begin{itemize}
        \item Utiliza la consulta MDX proporcionada.
        \item Asegúrate de que la conexión al cubo \textit{Finanzas} esté correctamente establecida.
    \end{itemize}
    \item \textbf{Crear los Gráficos en el Cuadro de Mando:}
    \begin{itemize}
        \item \textbf{Gráfico de Barras Horizontales (Gráfico 1):}
        \begin{itemize}
            \item Selecciona este tipo de gráfico en tu herramienta de dashboard.
            \item Asigna el ROI Promedio al eje X y las combinaciones al eje Y.
            \item Ordena las barras de mayor a menor ROI.
        \end{itemize}
        \item \textbf{Gráfico de Dispersión (Gráfico 2):}
        \begin{itemize}
            \item Elabora una nueva consulta MDX para obtener el número de proyectos por combinación.
            \begin{lstlisting}[style=terminal]
WITH 
MEMBER [Measures].[Numero de Proyectos] AS 
  [Measures].[Cantidad Proyectos]

MEMBER [Measures].[ROI Promedio] AS 
  AVG(
    [Tiempo].[Ano].Members,
    [Measures].[ROI]
  )

SELECT 
  {[Measures].[Numero de Proyectos], [Measures].[ROI Promedio]} ON COLUMNS,
  NONEMPTY(
    CROSSJOIN(
      [Director].[Nombre].[Nombre].Members,
      [Productor].[Nombre].[Nombre].Members
    )
  ) ON ROWS
FROM [Finanzas]
ORDER BY [Measures].[Numero de Proyectos] DESC
            \end{lstlisting}
            \item Nota: Asegúrate de tener definida la medida \texttt{[Measures].[Cantidad Proyectos]} que cuenta el número de proyectos realizados por cada combinación.
        \end{itemize}
    \end{itemize}
    \item \textbf{Personalizar y Configurar Detalles:}
    \begin{itemize}
        \item \textbf{Etiquetas y Leyendas:} Asegura que sean claras y descriptivas.
        \item \textbf{Colores y Estética:} Utiliza una paleta coherente que facilite la interpretación.
        \item \textbf{Interactividad:} Configura las opciones mencionadas en cada gráfico.
    \end{itemize}
\end{enumerate}

\subsection{4. Consideraciones Técnicas}

\begin{itemize}
    \item \textbf{Optimización de Consultas:}
    \begin{itemize}
        \item Si el volumen de datos es grande, considera limitar inicialmente el número de combinaciones mostradas o implementar paginación.
    \end{itemize}
    \item \textbf{Manejo de Datos Atípicos:}
    \begin{itemize}
        \item En el gráfico de dispersión, ten en cuenta las combinaciones con ROI extremadamente alto pero pocos proyectos, que pueden distorsionar la escala.
    \end{itemize}
    \item \textbf{Validación de Datos:}
    \begin{itemize}
        \item Asegura que los datos utilizados son correctos y actualizados.
    \end{itemize}
\end{itemize}

\subsection{5. Beneficios de Implementar Dos Gráficos}

\begin{itemize}
    \item \textbf{Perspectivas Complementarias:} Los dos gráficos ofrecen diferentes visiones del mismo conjunto de datos, enriqueciendo el análisis.
    \item \textbf{Identificación de Oportunidades Ocultas:} El gráfico de dispersión puede revelar combinaciones prometedoras que no destacan en el ranking tradicional.
    \item \textbf{Mejora de la Toma de Decisiones:} Al disponer de más información visual, se facilita la identificación de tendencias y patrones clave.
\end{itemize}

\subsection{6. Ejemplo de Visualización}

\textbf{Gráfico 1: Ranking de Combinaciones por ROI Promedio}

\begin{itemize}
    \item \textbf{Título:} ``Top Combinaciones de Directores y Productores por ROI Promedio''
    \item \textbf{Características Visuales:}
    \begin{itemize}
        \item Barras horizontales ordenadas de mayor a menor.
        \item Colores que representen diferentes rangos de ROI.
    \end{itemize}
\end{itemize}

\textbf{Gráfico 2: Relación entre Número de Proyectos y ROI Promedio}

\begin{itemize}
    \item \textbf{Título:} ``Relación entre Experiencia Conjunta y Rentabilidad''
    \item \textbf{Características Visuales:}
    \begin{itemize}
        \item Puntos dispersos en el plano según su número de proyectos y ROI.
        \item Posibilidad de diferenciar puntos por tamaño o color según otra métrica (por ejemplo, satisfacción de usuarios).
    \end{itemize}
\end{itemize}

\subsection{7. Conclusión}

Desarrollar esta funcionalidad en tu cuadro de mando proporcionará a la productora una herramienta poderosa para analizar y potenciar las colaboraciones entre directores y productores. Al basar las decisiones en datos concretos de ROI, se maximiza la probabilidad de éxito financiero en futuros proyectos.

\end{document}
