\documentclass[11pt]{opticajnl}
\journal{opticajournal} % use for journal or Optica Open submissions

% See template introduction for guidance on setting shortarticle option
\setboolean{shortarticle}{true}
% true = letter/tutorial
% false = research/review article

% ONLY applicable for journal submission shortarticle types:
% When \setboolean{shortarticle}{true}
% then \setboolean{memo}{true} will print "Memorandum" on title page header
% Otherwise header will remain as "Letter"
% \setboolean{memo}{true}
\usepackage{lineno}
\usepackage{listings}
\usepackage{xcolor}  % Para colorear el código

% Definir colores para la sintaxis de Python
\definecolor{pystring}{RGB}{186,33,33}      % Strings
\definecolor{pycomment}{RGB}{107,107,107}   % Comentarios
\definecolor{pykeyword}{RGB}{0,119,170}     % Palabras clave
\definecolor{pybuiltin}{RGB}{163,21,21}     % Funciones builtin
\definecolor{pybackground}{RGB}{250,250,250} % Fondo

% Definir el estilo para código Python
\lstdefinestyle{Python}{
    language=Python,
    backgroundcolor=\color{pybackground},
    basicstyle=\ttfamily\small,
    breakatwhitespace=false,
    breaklines=true,
    captionpos=b,
    keepspaces=true,
    numbers=left,
    numbersep=5pt,
    showspaces=false,
    showstringspaces=false,
    showtabs=false,
    tabsize=4,
    frame=single,
    commentstyle=\color{pycomment},
    keywordstyle=\color{pykeyword},
    stringstyle=\color{pystring},
    identifierstyle=\color{black},
    numberstyle=\tiny\color{gray},
    emphstyle=\color{pybuiltin},
    morekeywords={import,from,as,def,class,return,yield,for,while,if,else,elif,
                  try,except,finally,with,lambda,assert,pass,break,continue,
                  raise,global,nonlocal,True,False,None,and,or,not,is,in},
    emph={pandas,numpy,matplotlib,seaborn,sklearn,tensorflow,torch,
          plt,pd,np,sns,print,len,range,enumerate,zip,dict,list,tuple,set,
          min,max,sum,sorted,map,filter},
}

\lstdefinestyle{sql}{
  language=SQL,
  showspaces=false, 
  basicstyle=\ttfamily,
  numbers=left,
  numberstyle=\tiny,
  backgroundcolor=\color{gray!10},
  keywordstyle=\color{blue}\bfseries,
  commentstyle=\color{green!70}\ttfamily,
  stringstyle=\color{red!70}\ttfamily,
  breaklines=true,
  morekeywords={CREATE, DROP, TABLE, SELECT, INSERT, UPDATE, DELETE, FROM, WHERE, JOIN, ON, AS}, % Puedes añadir más palabras clave específicas de SQL.
  literate={``}{``}1 {''}{''}1 {“}{``}1 {”}{''}1 {‘}{`}1 {’}{'}1
}

\lstdefinestyle{terminal}{
  backgroundcolor=\color{white},   % Fondo blanco
  basicstyle=\color{black}\ttfamily, % Texto negro en fuente monoespaciada
  keywordstyle=\color{blue}\bfseries, % Palabras clave en azul y negrita
  commentstyle=\color{green!70}\ttfamily, % Comentarios en verde
  stringstyle=\color{red}\ttfamily, % Cadenas en rojo
  morekeywords={sudo, apt-get, install, cd, ls, mkdir, rm, rmdir, cp, mv, echo, cat, nano, vim, grep, find, chmod, chown, systemctl, service, update, upgrade, reboot, shutdown, exit}, % Comandos comunes de terminal
  breaklines=true, % Permitir saltos de línea
  frame=single, % Marco alrededor del código
  framerule=0.5pt, % Grosor del marco
  rulecolor=\color{gray}, % Color del marco
  xleftmargin=0.05\textwidth, % Margen izquierdo
  xrightmargin=0.05\textwidth, % Margen derecho
  aboveskip=1em, % Espacio antes del bloque de código
  belowskip=1em % Espacio después del bloque de código
}
\definecolor{rstring}{RGB}{186,33,33}     % Strings
\definecolor{rcomment}{RGB}{0,128,0}      % Comentarios
\definecolor{rfunction}{RGB}{0,0,255}      % Funciones
\definecolor{rkeyword}{RGB}{145,0,145}    % Palabras clave
\definecolor{rbackground}{RGB}{248,248,248} % Fondo

\lstdefinestyle{R}{
    language=R,
    backgroundcolor=\color{rbackground},
    basicstyle=\ttfamily\small,
    breakatwhitespace=false,
    breaklines=true,
    captionpos=b,
    keepspaces=true,
    numbers=left,
    numbersep=5pt,
    showspaces=false,
    showstringspaces=false,
    showtabs=false,
    tabsize=2,
    frame=single,
    commentstyle=\color{rcomment},
    keywordstyle=\color{rkeyword},
    stringstyle=\color{rstring},
    identifierstyle=\color{black},
    numberstyle=\tiny\color{gray},
    morekeywords={library, data.frame, read.csv, ggplot, aes, geom_bar, theme_minimal, 
                  scale_fill_manual, gather, group_by, summarise, arrange, filter},
}


%\linenumbers % Turn off line numbering for Optica Open preprint submissions.

\title{Consultas MDX}

\author[1,2,3]{Luis Ardévol Mesa}


\begin{abstract}
A
\end{abstract}

\setboolean{displaycopyright}{false} % Do not include copyright or licensing information in submission.

\begin{document}

\maketitle

\section{Identificar las combinaciones de director y productor con mayor rentabilidad, calculando el retorno de inversión (ROI) promedio de sus proyectos.}

Esta consulta permite a las productoras identificar dúos creativos que generan altos retornos de inversión, lo cual es valioso para futuras colaboraciones y asignación de recursos. Identificación de colaboraciones rentables.

\begin{lstlisting}[style=terminal]
WITH MEMBER [Measures].[ROI Promedio] AS
AVG(
  [Tiempo].[Tiempo].Members,
  ( [Measures].[Ingresos] - [Measures].[Coste] ) / NULLIF( [Measures].[Coste], 0 )
)

SELECT 
  {[Measures].[ROI Promedio]} ON COLUMNS,
  ORDER(
    NONEMPTY(
      [Director].[Nombre].[Nombre].Members * [Productor].[Nombre].[Nombre].Members
    ),
    [Measures].[ROI Promedio],
    BDESC
  ) ON ROWS
FROM [Finanzas]
\end{lstlisting}


\section{Determinar la tendencia de satisfacción de usuarios por productora en los últimos 5 años, identificando si va en aumento o descenso.}

Esta consulta ayuda a las productoras a evaluar cómo percibe el público su contenido a lo largo del tiempo, permitiendo tomar acciones para mejorar o mantener la satisfacción.


\begin{lstlisting}[style=terminal]
WITH MEMBER [Measures].[Tendencia Satisfaccion] AS
LinRegSlope(
  LastPeriods(5, [Tiempo Votacion].[Ano].CurrentMember), 
  [Tiempo Votacion].[Ano].CurrentMember.MemberValue, 
  [Measures].[Satisfaccion]
)

SELECT 
  {[Measures].[Tendencia Satisfaccion]} ON COLUMNS,
  [Productora].[Nombre].[Nombre].Members ON ROWS
FROM [Satisfaccion Usuarios]
\end{lstlisting}



\section{Comparar el porcentaje de mercado en ingresos de cada productora por año.}

Permite identificar la cuota de mercado de las productoras en términos de ingresos, ayudando a entender su posición competitiva en la industria.


\begin{lstlisting}[style=terminal]
WITH MEMBER [Measures].[Ingresos Totales Industria] AS
SUM(
  [Productora].[Nombre].Members, 
  [Measures].[Ingresos]
)

MEMBER [Measures].[Cuota de Mercado] AS
([Measures].[Ingresos] / [Measures].[Ingresos Totales Industria]) * 100

SELECT 
  {[Measures].[Cuota de Mercado]} ON COLUMNS,
  NONEMPTY(
    [Productora].[Nombre].[Nombre].Members *
    [Tiempo].[Ano].[Ano].Members
  ) ON ROWS
FROM [Finanzas]
\end{lstlisting}



\section{Identificar los meses en los que las productoras tienen picos de ingresos y analizar patrones estacionales.}

Esta consulta es útil para planificar lanzamientos futuros en los meses con mayor potencial de ingresos, optimizando así el rendimiento financiero.


\begin{lstlisting}[style=terminal] 
SELECT 
  {[Measures].[Ingresos]} ON COLUMNS,
  NONEMPTY(
    [Productora].[Nombre].[Nombre].Members *
    [Tiempo].[Mes].[Mes].Members
  ) ON ROWS
FROM [Finanzas]
ORDER BY [Measures].[Ingresos] DESC
\end{lstlisting}


\section{Evaluar el impacto de los votos de usuarios en los ingresos, analizando si mayores votos se correlacionan con mayores ingresos para cada director.}

Esto ayuda a entender la relación entre la participación de la audiencia y el desempeño financiero, lo que puede influir en estrategias de marketing y engagement.


\begin{lstlisting}[style=terminal]
SELECT 
  {[Measures].[Ingresos], [Measures].[Votos]} ON COLUMNS,
  NONEMPTY(
    [Director].[Nombre].[Nombre].Members
  ) ON ROWS
FROM [Finanzas]
INNER JOIN [Satisfaccion Usuarios]
ON (
  [Finanzas].[Director].CurrentMember = [Satisfaccion Usuarios].[Director].CurrentMember
)
\end{lstlisting}

\end{document}
