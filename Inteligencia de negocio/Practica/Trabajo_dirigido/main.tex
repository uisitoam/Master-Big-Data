\documentclass[12pt]{opticajnl}
\journal{opticajournal} 

\setboolean{shortarticle}{true}


\usepackage{lineno}
\usepackage{setspace} % interlineado
\usepackage{tabularray}
\usepackage{multicol}

%\linenumbers % Turn off line numbering for Optica Open preprint submissions.

\title{El Guachinche}

%\author[1,2,3]{Luis Ardévol Mesa, Carlos Martínez García \\ \copyrightstatement}
\author[1,2,3]{Luis Ardévol Mesa, Carlos Martínez García}

\begin{document}

\maketitle

\section{Introducción}

\spacing{1.1}

Los \textit{guachinches} son establecimientos (en la propia vivienda del propietario) que surgen en la zona del norte del Tenerife como solución para vender el excedente de vino de cada cosecha. El vino se acompaña típicamente con platos tradicionales de la cocina canaria. Actualmente, según el Decreto 83/2013\footnote{\url{https://www.gobiernodecanarias.org/boc/2013/153/001.html}}, hay una regulación muy estricta amparando a estos locales y al viticultor; muchos establecimientos comúnmente denominados ``guachinches'', entran dentro de la categoría \textit{restaurante}, al salirse de lo establecido en este decreto. \\

El Guachinche nace de una idea sencilla pero atractiva: traer el espíritu de los tradicionales guachinches canarios al resto de España, pero con un toque especial. En nuestra cocina, las recetas isleñas comparten protagonismo con una cuidada selección de arepas para todos los públicos, todo esto acompañado de los mejores vinos de las Islas. Es ese equilibrio entre tradición y adaptación lo que nos hace únicos: cada local incorpora platos de la región, porque creemos que la gastronomía local es parte fundamental de nuestra historia. \\

No obstante, quien entra a El Guachinche no solo viene a comer. Para dar una experiencia lo más cercana posible y crear una atmósfera atractiva y familiar para los comensales, nuestros locales presentan una ambientación tradicional y hogareña. La experiencia no sería completa de no ser por nuestro personal, siempre cercano. \\

Actualmente, El Guachinche cuenta con 13 locales, repartidos en Canarias (7), Galicia (2), Andalucía (2) y Comunidad Valenciana (2). El crecimiento en los últimos dos años ha sido exponencial, por lo que desde la dirección de la empresa nos vemos en la necesidad de implementar un modelo de inteligencia de negocio que nos permita tomar decisiones basadas en datos, priorizando siempre al cliente. Algunos de los objetivos perseguidos serían la fidelización de clientes y la mejora de la experiencia general de los mismos, adaptar la oferta gastronómica a las preferencias de cada región o mantener un margen de beneficio cómodo en cada uno de los locales. \\


\subsection{Áreas de negocio}
Nuestra empresa se estructura en cinco áreas fundamentales, lo que nos permite abarcar todos los aspectos necesarios para mantener el correcto funcionamiento de los locales y asegurar su éxito a largo plazo:

\begin{itemize}
    \item \textbf{Área de gestión y finanzas:} Este área se preocupa por la rentabilidad del negocio, para lo cual es necesaria una gestión administrativa y financiera. Controla los costes, ingresos y presupuesto, además de establecer estrategias para optimizar la rentabilidad de cada local.
    \item \textbf{Área de expansión:} La apertura de un nuevo local nunca es tarea fácil. Para ello, disponemos del área de expansión, que estudia el mercado en busca de nuevas oportunidades de crecimiento en ubicaciones aún no exploradas por El Guachinche.
    \item \textbf{Área de gestión de clientes:} Nuestra prioridad es la satisfacción del cliente. Este área se ocupa de garantizar la mejor experiencia posible en cada uno de nuestros establecimientos.
    \item \textbf{Área de producción y operaciones:} Gestionar suministros de calidad en todos nuestros locales es una tarea complicada. El área de producción y operaciones se encarga de coordinar la cadena de suministro, tanto de productos canarios como locales. Esto permite mantener unos estándares de calidad y controlar el \textit{stock}.
    \item \textbf{Área comercial:} Desarrolla las estrategias de mercado, adaptando precios y promociones a cada zona mientras mantiene la coherencia de marca en todos los establecimientos.
\end{itemize}













\section{Objetivos de negocio}

\subsection{Área de gestión y finanzas}

\subsubsection*{CSF1: Garantizar la rentabilidad de cada local.}

La rentabilidad de los locales es uno de los objetivos más importantes de cualquier negocio. A través de la gestión minuciosa de los ingresos y los costos operativos, podemos asegurar que cada local mantenga un rendimiento económico sostenible, lo que garantizará parte del éxito financiero de la cadena. Para lograr este objetivo, se utilizan los siguientes indicadores:

\begin{itemize}
    \item \textbf{PI1.1: Fijar margen de beneficio bruto en un 70\%.} Este indicador mide los beneficios obtenidos tras considerar los costes directos asociados. El objetivo será que cada local fije este margen en torno a un 65-70\%, lo que aseguraría cubrir los gastos del local y permitiría un margen de beneficio neto del 10-15\%. 
    \item \textbf{PI1.2: Disminuir coste operativo por local en un 10\%.} Dentro de estos costes se incluyen el alquiler, el salario del personal, los suministros y gastos extra, como luz, agua, limpieza, mantenimiento, etc. La meta es reducir estos gastos en un 10\% en cada local sin comprometer la calidad del servicio, lo que permitirá aumentar la rentabilidad.
    \item \textbf{PI1.3: Aumentar el promedio de gasto por cliente en un 5\%.} Este indicador mide los ingresos promedio generados por cada cliente. El objetivo sería aumentarlo sin elevar siginificativamente el coste de cada plato, lo que indicaría un mayor consumo por cliente. 
    \item \textbf{PI1.4: Fijar tasa de ocupación del local en un 80\%.} Para asegurar la rentabilidad de cada local, es crucial aprovechar de forma eficiente el espacio durante el servicio. Mantener un alto porcentaje de las mesas disponibles ocupadas es un indicador clave de la rentabilidad del local.
    \item \textbf{PI1.5: Reducir porcentaje de desperdicio de alimentos a un 2\%.} El desperdicio de alimentos es inaceptable en nuestros locales, no solo por las consecuencias económicas. Se exige mantenerlo por debajo del 2\% para asegurar la rentabilidad de los productos. 
\end{itemize}






\subsection{Área de expansión}

\subsubsection*{CSF2: Mantener un crecimiento escalado y sostenible.}

Mantener la esencia del negocio es de vital importancia. Por ello, queremos garantizar que un crecimiento de la cadena sostenible, que no comprometa la calidad del servicio y la experiencia del cliente. Para cumplir este objetivo, fijamos los siguientes indicadores:

\begin{itemize}
    \item \textbf{PI2.1: Bajo número de nuevos locales abiertos anualmente.} Una vez establecido el negocio, no existe una necesidad de expansión rápida y descontrolada. El número de nuevos locales abiertos cada año será de máximo 3. Esto garantiza una correcta gestión de las finanzas y la logística de los mismos, haciendo que, al abrir nuevos locales, los ya establecidos se encuentren en una situación financiera estable.
    \item \textbf{PI2.2: Reducir tiempo promedio de rentabilidad de nuevos locales.} Este indicador va de la mano con el anterior. Reducir el tiempo que un nuevo local tarda en ser rentable a menos de 12 meses hace posible la apertura anual de nuevos locales sin aumentar en exceso la logística.
    \item \textbf{PI2.3: Incremento en ingresos totales del 15\%.} Buscamos un crecimiento sostenido del negocio. Por ello, un crecimiento anual del 10-15\% resulta adecuado.
    \item \textbf{PI2.4: Tasa de éxito de nuevos locales de un 90\%.} El éxito de los nuevos locales se mide en base a la satisfacción de sus clientes y el alcance de objetivos (como la rentabilidad del primer año). Así, buscamos fijar la tasa de éxito de los nuevos locales en no menos de un 90\%.  
\end{itemize}






\subsection{Área de gestión de clientes}

\subsubsection*{CSF3: Maximizar la satisfacción y fidelización del cliente.}

El cliente es el centro de la estrategia de El Guachinche. Aumentar la satisfacción del cliente y fomentar la fidelización es esencial para el éxito a largo plazo de la cadena. A través de la atención personalizada y la calidad y cercanía del servicio, se busca crear una experiencia única para el cliente. Los indicadores clave en este área son:

\begin{itemize}
    \item \textbf{PI3.1: Valoración de satisfacción del cliente de $4.5$.} Nuestros locales realizan encuestas de satisfacción recurrentes a sus clientes, con el fin de obtener \textit{feedback} acerca de la calidad de la comida, el servicio y el ambiente del local. Mantener una satisfacción por encima del $4.5/5$ en estas encuestas es un objetivo principal para el éxito de la cadena.
    \item \textbf{PI3.2: Aumentar la recuencia de visitas por cliente.} El objetivo es que nuestros clientes habituales visiten el local al menos 2 veces al mes. Esto indica un nivel de satisfacción adecuado y permite apoyar la economía del local en una clientela ya estable.
    \item \textbf{PI3.3: 90\% de quejas resueltas satisfactoriamente.} Si bien no es deseable contar con un gran número de quejas, estas forman parte de un negocio. En este aspecto, el objetivo es resolver de forma satisfactoria al menos un 90\% de las mismas, lo que ayudará a mejorar la imagen de cara al público y la retención de clientes.
    \item \textbf{PI3.4: Aumentar los clientes nuevos por recomendación.} No hay mejor publicidad que la proporcionada por el propio cliente boca a boca. Este método, además, se alinea a la perfección con la filosofía tradicional de nuestra cadena. En cuanto a la captación de nuevos clientes, el objetivo sería que al menos un 80\% de los nuevos clientes vengan por recomendación, lo que indicaría que los clientes actuales están promocionando de forma activa el negocio.
\end{itemize}

\subsection{Área de producción y operaciones}

\textbf{CSF4: Eficiencia en la gestión de la cadena de suministro.}

La cadena de suministro es crucial para garantizar que los productos estén siempre disponibles, frescos y de la mejor calidad. Un manejo eficiente del inventario, logística y reposición de productos es vital para evitar rupturas de \textit{stock} y sobrecostos. Como indicadores clave en este área se plantean los siguientes:

\begin{itemize}
    \item \textbf{PI4.1: Tiempo de reposición de productos menor a 48 horas.} Con esto se hace referencia al tiempo trancurrido entre un pedido y la entrega del mismo en el local. Para que estos últimos dispongan de productos frescos y disponibles, debemos asegurar que este tiempo de reposición sea de 24 horas para productos locales y menor a 48 horas para los productos que vengan desde Canarias. 
    \item \textbf{PI4.2: Nivel de inventario crítico por debajo del 5\%.} El inventario crítico hace referencia a aquellos productos que, por su poca cantidad disponible, están en riegso de ruptura. Para disponer siempre de suficiente inventario para el día a día, buscamos fijar el nivel de inventario crítico en menos de un 5\%. No obstante, se busca un compromiso para evitar el exceso de \textit{stock} (lo que ayuda a reducir el desperdicio de alimentos).
    \item \textbf{PI4.3: Reducir el coste logístico por pedido en un 10\%.} Para reducir los costes asociados a cada pedido de suministros, se busca depender todo lo posible de productos de kilómetro 0. El objetivo final es reducir los costes actuales en un 5-10\%.
\end{itemize}

\textbf{CSF5: Flexibilidad operativa.}

En unos locales con la demanda que presenta El Guachinche, es necesario mantener cierta flexibilidad operativa que permita resolver problemas inesperados de forma que afecte lo mínimo posible al cliente. Para mejorar nuestra capacidad de ajuste a la demanda y condiciones del mercado, definimos los siguientes indicadores:

\begin{itemize}
    \item \textbf{PI5.1: Menú adaptable de forma parcial.} Buscamos que al menos un 15\% de la carta de cada local sea variable en función de la estacionalidad de los productos. 
    \item \textbf{PI5.2: Disponibilidad de 2 proveedores alternativos.} Para evitar cortes de suministros y poder disponer siempre de productos frescos en los locales, resulta necesario que cada local tenga al menos dos proveedores de calidad para momentos críticos.
    \item \textbf{PI5.3: Tiempo de parada por falta de suministro, 0.} Con la adapatabilidad del menú y la exigencia de proveedores alternativos, buscamos que no haya cortes o reducciones en las horas de servicios de cada local.    
\end{itemize}





\subsection{Área comercial}

\subsubsection*{CSF6: Coherencia de la marca.}

Como se puede intuir por la descripción e idea del negocio, buscamos una imagen de marca reconocible y coherente con los valores que representamos. Al ser una cadena nacional, la coherencia cobra un papel de gran relevancia. Para garantizar esta última, fijamos los siguientes indicadores:

\begin{itemize}
    \item \textbf{PI6.1: Cumplimiento de estándares de marca en todos los locales.} El 100\% de los locales deben cumplir con las directrices de imagen y servicio fijadas por la directiva de forma transversal.
    \item \textbf{PI6.2: Comunicación y publicidad consistentes.} Las campañas de publicidad de cada local deben estar alineadas con los valores que representa la empresa. El Guachinche se debe ver como un todo y no como locales individuales.
    \item \textbf{PI6.3: Formación del 100\% del personal.} Todo el personal debe comprender los valores que representa la cadena y conocer la cultura y tradición local. En este sentido, el 100\% de los trabajadores deberá recibir cierta formación.
\end{itemize}

A continuación, se incluye una tabla a modo de síntesis de los factores críticos de éxito y los indicadores clave de rendimiento que se han explicado:


\begin{longtblr}[caption = {Síntesis factores críticos de éxito.},]{colspec = {p{1.75cm}p{2.5cm}Xp{1.9cm}p{1.5cm}X}, rowhead = 1,}
\hline\hline
\textbf{Factor Crítico de Éxito (CSF)} & \textbf{Indentificador de indicador} & \textbf{Indicador} & \textbf{Tipo de indicador} & \textbf{Meta} & \textbf{Acción} \\ \hline\hline
CSF1 & PI1.1 & Aumentar margen de beneficio bruto & \textit{leading} & 65-70\% &  \\ \hline
CSF1 & PI1.2 & Reducir coste operativo por local & \textit{lagging} & $-10$\% &  \\ \hline
CSF1 & PI1.3 & Aumentar gasto por cliente & & +5-10\% & \\ \hline
CSF1 & PI1.4 & Aumentar tasa de ocupación & & 75-80\% & \\ \hline
CSF1 & PI1.5 & Reducir desperdicio de alimentos & & $<2$\% & \\ \hline\hline
CSF2 & PI2.1 & Pocos nuevos locales por año & & $\leq 3$ & \\ \hline
CSF2 & PI2.2 & Reducir tiempo de rentabilidad de nuevos locales & & $< 12$ meses & \\ \hline
CSF2 & PI2.3 & Aumento de ingresos totales & & $+15$\% & \\ \hline
CSF2 & PI2.4 & Fijar tasa de éxito de nuevos locales & & $\geq 90$\% & \\ \hline\hline
CSF3 & PI3.1 & Fijar valoración de satisfacción del cleinte & & $\geq 4.5$ & \\ \hline
CSF3 & PI3.2 & Aumentar recurrencia de visitas & & $\geq 2$/mes & \\ \hline
CSF3 & PI3.3 & Fijar quejas resultas satisfactoriamente & & 90\% & \\ \hline
CSF3 & PI3.4 & Aumentar nuevos clientes por recomendación & & $\geq 80$\% & \\ \hline\hline
CSF4 & PI4.1 & Fijar tiempo de reposición de productos & & $< 48$h & \\ \hline
CSF4 & PI4.2 & Reducir nivel de inventario crítico & & $< 5$\% & \\ \hline
CSF4 & PI4.3 & Reducir costes logísticos por pedido & & $-10$\% & \\ \hline\hline
CSF5 & PI5.1 & Adaptabilidad del menú & & 15\% & \\ \hline
CSF5 & PI5.2 & Tener proveedores alternativos & & 2 & \\ \hline
CSF5 & PI5.3 & Reducir el tiempo de parada por falta de suministros & & 0 & \\ \hline\hline
CSF6 & PI6.1 & Cumplimiento de estándares de la marca por los locales & & 100\% & \\ \hline
CSF6 & PI6.2 & \textit{Marketing} consistente & & Uniformidad & \\ \hline
CSF6 & PI6.3 & Formar al personal en cultura y tradición local & & 100\% del personal & \\ \hline\hline
\end{longtblr}



\newpage

\section{Diseño lógico}

\subsection{Procesos de negocio}

Los principales procesos de negocio identificados son los siguientes:

\textbf{Proceso 1:} Gestión de inventario y cadena de suministro.  
\textbf{Granularidad:} Producto por local.  
\textbf{Medidas:} Tiempo de reposición, coste logístico.  
\textbf{Tasa de refresco:} Semanal.  
\textbf{Tipo de tabla de hechos:} Hecho acumulativo.

\textbf{Proceso 2:} Gestión financiera.  
\textbf{Granularidad:} Ingresos y gastos por local.  
\textbf{Medidas:} Beneficio neto, coste operativo.  
\textbf{Tasa de refresco:} Mensual.  
\textbf{Tipo de tabla de hechos:} Hecho periódico.

\textbf{Proceso 3:} Gestión de clientes.  
\textbf{Granularidad:} Cliente por transacción.  
\textbf{Medidas:} Frecuencia de visitas, satisfacción del cliente.  
\textbf{Tasa de refresco:} Diaria.  
\textbf{Tipo de tabla de hechos:} Hecho transaccional.

\subsection{Dimensiones}
Las principales dimensiones identificadas incluyen:

\textbf{Dimensión de Producto:} 
\begin{itemize}
    \item Atributos: Nombre del producto, categoría, precio.
    \item Compartida entre los procesos de producción y ventas.
\end{itemize}

\textbf{Dimensión de Cliente:} 
\begin{itemize}
    \item Atributos: ID de cliente, nombre, edad, preferencias de compra.
    \item Compartida entre el proceso de gestión de clientes y ventas.
\end{itemize}

\subsection{Cálculo de indicadores}

La siguiente tabla muestra el cálculo de algunos de los PIs definidos a partir de los procesos de negocio identificados:

\begin{table}[H]
\centering
\begin{tabular}{|p{2cm}|p{4cm}|p{4cm}|p{7cm}|}
\hline
\textbf{Identificador del PI} & \textbf{Indicador} & \textbf{Proceso de negocio} & \textbf{Descripción del cálculo} \\ \hline
PI1                          & Margen bruto       & Gestión financiera          & (Ingresos - Costes) / Ingresos \\ \hline
PI2                          & Coste operativo    & Gestión financiera          & Suma de los costes operativos \\ \hline
PI3                          & Beneficio neto     & Gestión financiera          & Ingresos - Costes \\ \hline
PI4                          & Tiempo de reposición & Cadena de suministro       & Fecha de pedido - Fecha de entrega \\ \hline
PI5                          & Nivel de inventario crítico & Cadena de suministro & Inventario en niveles críticos \\ \hline
PI6                          & Coste logístico    & Cadena de suministro        & Costes asociados a la logística \\ \hline
PI7                          & Satisfacción del cliente & Gestión de clientes     & Promedio de puntuaciones de encuestas \\ \hline
PI8                          & Frecuencia de visitas & Gestión de clientes       & Número de visitas / cliente / mes \\ \hline
PI9                          & Tasa de retención  & Gestión de clientes         & Porcentaje de clientes recurrentes \\ \hline
\end{tabular}
\caption{Cálculo de indicadores}
\end{table}




%%%%%%%%%%%%%


\section{Diseño y construcción del almacén de datos}

Nuestros locales llevan un registro cuidadoso de los aspectos más importantes de la actividad diaria, como son las ventas, los gastos o la opinion de los clientes. Cada local genera reportes mensuales que recogen estos datos y son enviados a la central para su análisis. Estos reportes son los que nos permitirán obtener la información necesaria para la toma de decisiones. 

\subsection{Diseño del almacén de datos}

Para realizar el diseño del almacén de datos, es necesario conocer primero la estructura de los datos recogidos en los reportes de cada local. Concretamente, cada mes se reciben dos archivos de datos por cada local de nuestra cadena. El primero de ellos contiene toda la información relacionada con la economía del restaurante: desglose de gastos en cuatro categorías (personal, suministros, alquiler y otros), ingresos por ventas (diferenciando entre los servicios presencial y domicilio) y clientela. Esto será beneficio para varios de los objetivos de la cadena. Siendo un poco más específicos, se proporcionan los siguientes datos:
\begin{itemize}
\item Un identificador entero para cada local, generado por la central en orden de apertura.
\item La fecha de envío del reporte (establecida para el primer día del mes). 
\item Los gastos en alquiler, personal, suministros (proveedores), y extras (limpieza, mantenimiento, etc.), todo en euros.
\item Los ingresos, tanto por vía presencial como a domicilio, en euros. Como ya se ha mencionado, estos ingresos ya tienen descontada la comisión de Uber Eats o Glovo.
\item Los clientes totales por ambos canales, así como el número de clientes nuevos.
\item Los platos ofertados en el local, cada uno con los siguientes datos:
\begin{itemize}
\item Un identificador entero para cada plato, generado por la central.
\item El nombre, precio y número de ventas del plato. 
\end{itemize}
\end{itemize}

Estos datos se recogen en un archivo XML como el que se muestra en la figura \ref{fig:xml_economico}.

\begin{figure}[h]
\centering
\includegraphics[width=0.6\textwidth]{fotos/1.png}
\caption{Ejemplo de archivo XML con los datos económicos mensuales de un local.}
\label{fig:xml_economico}
\end{figure}

\noindent Los identificadores de los locales son los siguientes:
\begin{multicols}{2}
\begin{itemize}
\item 1: La Laguna, Tenerife.
\item 2: Hermigua, La Gomera.
\item 3: Cotillo, Fuerteventura.
\item 4: Las Palmas de Gran Canaria, Gran Canaria.
\item 5: Tazacorte, La Palma.
\item 6: Valverde, El Hierro.
\item 7: Arrecife, Lanzarote.
\item 8: Granada, Granada.
\item 9: Sevilla, Sevilla.
\item 10: Santiago de Compostela, A Coruña.
\item 11: Vigo, Pontevedra.
\item 12: Alicante, Comunidad Valenciana.
\item 13: Valencia, Comunidad Valenciana. 
\end{itemize}
\end{multicols}

\noindent Los identificadores de los productos son los siguientes:
\begin{multicols}{3}
\begin{itemize}
\item 1: Almogrote Gomero.
\item 2: Papas arrugadas con mojo.
\item 3: Queso asado con mojo.
\item 4: Escaldón.
\item 5: Ropa vieja.
\item 6: Costilla con papas y piña.
\item 7: Carne fiesta.
\item 8: Quesillo.
\item 9: Bienmesabe.
\item 10: Arepa reina pepiada.
\item 11: Arepa pabellón.
\item 12: Arepa full equipo.
\item 13: Arepa vegana.
\item 14: Arepa blanca.
\item 15: Vino tinto canario.
\item 16: Vino blanco canario.
\item 17: Agua.
\item 18: Refresco de cola.
\item 19: Refresco de limón.
\item 20: Refresco de naranja.
\item 21: Aquarius.
\item 22: Nestea mango-piña.
\item 23: Salmorejo.
\item 24: Pescaito frito.
\item 25: Gambitas de Huelva.
\item 26: Pestiños.
\item 27: Vino tinto andaluz.
\item 28: Vino blanco andaluz.
\item 29: Pulpo a feira.
\item 30: Empanada (porción).
\item 31: Lacon con grelos.
\item 32: Tarta de Santiago.
\item 33: Vino tinto gallego.
\item 34: Vino blanco gallego.
\item 35: Paella.
\item 36: Arroz negro.
\item 37: Esgarraet. 
\item 38: Fartons. 
\item 39: Vino tinto valenciano.
\item 40: Vino blanco valenciano.
\end{itemize}
\end{multicols}

Como bien describe el objetivo del negocio, cada región tiene sus platos típicos. Los 22 primeros productos son comunes a todos los locales de la cadena, mientras que a partir de ahí, cada región incorpora distintos productos, como es el caso del \textit{salmorejo} en Andalucía, el \textit{lacçon con grelos} en Galicia o el \textit{esgarraet} en la Comunidad Valenciana. Así mismo, cada local ofrece vinos de proximidad. \\

Para cumplir los objetivos relacionados con la satisfacción y fidelización de clientes, cada local proporciona un segundo archivo, en este caso en formato CSV, con valoraciones de los clientes: en caso de ser clientes presenciales, se valora el ambiente del local, el personal y la calidad de la comida, mientras que los clientes a domicilio simplemente hacen llegar una valoración del servicio general. Así, cada fila del archivo da el identificador del local, la fecha de la valoración, y las votaciones correspondientes (3 en el caso de clientes presenciales, una en el caso de clientes a domicilio). Un ejemplo de este archivo se muestra en la figura \ref{fig:csv_valoraciones}.

\begin{figure}[h]
\centering
\includegraphics[width=0.6\textwidth]{fotos/2.png}
\caption{Ejemplo de archivo CSV con las valoraciones de los clientes de un local.}
\label{fig:csv_valoraciones}
\end{figure}

Se dará a los datos almacenados de cada uno de los archivos anteriores la máxima granularidad posible, es decir, mensual. Las dimensiones seleccionadas son las siguientes:
\begin{itemize}
\item \textbf{Tiempo}: la dimensión temporal de los datos se representa mediante dos niveles: año y mes.
\item \textbf{Restaurante}: la dimensión restaurante actúa como dimensión geográfica. Se tendrán en cuenta todos los locales de la cadena y se mostrarán dos niveles, uno para el país y otro para la ciudad.
\item \textbf{Producto}: la dimensión producto recoge todos los platos ofertados en los locales de la cadena. Esta es una dimensión plana que muestra los productos y el precio asociado a cada uno de ellos.
\end{itemize}
\noindent Los hechos seleccionados para el almacén de datos son los siguientes:
\begin{itemize}
\item \textbf{Finanzas}: se almacenan los costes e ingresos de los locales, junto con los datos de los clientes para cada combinación de las siguientes dimensiones: tiempo (mes de envío de los reportes) y restaurante (mediante el identificador de cada local).
\item \textbf{Producto}: se almacenan los hechos relacionados con las ventas de los productos ofertados en los locales, para cada combinación de las siguientes dimensiones: tiempo (mes de envío de los reportes), restaurante (mediante el identificador de cada local) y producto (mediante el identificador de cada plato). Se tiene el númer ototal de ventas mensuales de un producto, así como un cálculo de los ingresos generados por ese producto, a partir de sus ventas y su precio.
\item \textbf{Feedback}: se almacenan las valoraciones de los clientes para cada combinación de las siguientes dimensiones: tiempo (mes de envío de los reportes) y restaurante (mediante el identificador de cada local). Se almacena la media de las valoraciones de los clientes para cada uno de los aspectos recogidos en los reportes.
\end{itemize}

Por tanto, el modelo conceptual del almacén de datos se muestra en la figura \ref{fig:esquema_almacen}, siendo un esquema con tres estrellas y tres dimensiones.

\begin{figure}[h]
\centering
\includegraphics[width=0.6\textwidth]{fotos/3.pdf}
\caption{Esquema conceptual del almacén de datos.}
\label{fig:esquema_almacen}
\end{figure}

Los detalles de la implementación del almacén de datos se describen en la siguiente tabla: 

\begin{table}[H]
\centering
\begin{tabular}{cll}
\hline
\multicolumn{3}{|c|}{Dimensiones} \\ \hline
\multicolumn{1}{l}{} & & \\ \hline
\multicolumn{3}{|c|}{Tiempo} \\ \hline \hline
\multicolumn{1}{|c}{Atributo} & \multicolumn{1}{c}{Tipo} & \multicolumn{1}{c|}{Descripción} \\ \hline
\multicolumn{1}{|l}{id (pk)} & Integer & \multicolumn{1}{l|}{\begin{tabular}[c]{@{}l@{}}Identificador autoincremental generado por el almacén \\ cada vez que se inserta un nuevo mes. \end{tabular}} \\ 
\multicolumn{1}{|l}{} &  & \multicolumn{1}{l|}{\begin{tabular}[c]{@{}l@{}}\end{tabular}} \\ \hline
\multicolumn{1}{|l}{ano} & Integer & \multicolumn{1}{l|}{\begin{tabular}[c]{@{}l@{}}Número correspondiente al año. \end{tabular}} \\ \hline
\multicolumn{1}{|l}{mes} & Integer & \multicolumn{1}{l|}{\begin{tabular}[c]{@{}l@{}}Número correspondiente al mes dentro del año. \end{tabular}} \\ \hline
\multicolumn{1}{|l}{mes\_texto} & String & \multicolumn{1}{l|}{\begin{tabular}[c]{@{}l@{}}Texto con el nombre del mes. \end{tabular}} \\ \hline
\multicolumn{1}{l}{} &  & \\
\end{tabular}
\end{table}
\begin{table}[H]
\centering
\begin{tabular}{cll}
\hline
\multicolumn{3}{|c|}{Restaurante} \\ \hline \hline
\multicolumn{1}{|c}{Atributo} & \multicolumn{1}{c}{Tipo} & \multicolumn{1}{c|}{Descripción} \\ \hline
\multicolumn{1}{|l}{id (pk)} & Integer & \multicolumn{1}{l|}{\begin{tabular}[c]{@{}l@{}}Identificador de cada local, generado por la central. \end{tabular}} \\ 
\multicolumn{1}{|l}{} &  & \multicolumn{1}{l|}{\begin{tabular}[c]{@{}l@{}}\end{tabular}} \\ \hline
\multicolumn{1}{|l}{pais} & String & \multicolumn{1}{l|}{\begin{tabular}[c]{@{}l@{}}País donde se sitúa el local. \end{tabular}} \\ \hline
\multicolumn{1}{|l}{ciudad} & String & \multicolumn{1}{l|}{\begin{tabular}[c]{@{}l@{}}Ciudad donde se sitúa el local. \end{tabular}} \\ \hline
\multicolumn{1}{l}{} &  & \\ \hline
\multicolumn{3}{|c|}{Producto} \\ \hline \hline
\multicolumn{1}{|c}{Atributo} & \multicolumn{1}{c}{Tipo} & \multicolumn{1}{c|}{Descripción} \\ \hline
\multicolumn{1}{|l}{id (pk)} & Integer & \multicolumn{1}{l|}{\begin{tabular}[c]{@{}l@{}}Identificador de cada plato, generado por la central. \end{tabular}} \\ \hline 
\multicolumn{1}{|l}{nombre} & String & \multicolumn{1}{l|}{\begin{tabular}[c]{@{}l@{}}Nombre del plato. \end{tabular}} \\ \hline
\multicolumn{1}{|l}{precio} & Numeric & \multicolumn{1}{l|}{\begin{tabular}[c]{@{}l@{}}Precio del plato. \end{tabular}} \\ \hline
\multicolumn{1}{l}{} &  & \\
\multicolumn{1}{l}{} &  & \\
\end{tabular}
\end{table}

\begin{table}[H]
\centering
\begin{tabular}{cll}
\hline
\multicolumn{3}{|c|}{Hechos} \\ \hline
\multicolumn{1}{l}{} & & \\ \hline
\multicolumn{3}{|c|}{Finanzas} \\ \hline \hline
\multicolumn{1}{|c}{Atributo} & \multicolumn{1}{c}{Tipo} & \multicolumn{1}{c|}{Descripción} \\ \hline
\multicolumn{1}{|l}{alquiler} & Numeric & \multicolumn{1}{l|}{\begin{tabular}[c]{@{}l@{}}Gasto en alquiler de un local, en euros. \end{tabular}} \\ \hline
\multicolumn{1}{|l}{personal} & Numeric & \multicolumn{1}{l|}{\begin{tabular}[c]{@{}l@{}}Gasto en personal de un local, en euros. \end{tabular}} \\ \hline
\multicolumn{1}{|l}{proveedores} & Numeric & \multicolumn{1}{l|}{\begin{tabular}[c]{@{}l@{}}Gasto en suministros de un local, en euros. \end{tabular}} \\ \hline
\multicolumn{1}{|l}{extra} & Numeric & \multicolumn{1}{l|}{\begin{tabular}[c]{@{}l@{}}Gasto en extras de un local, en euros. \end{tabular}} \\ \hline
\multicolumn{1}{|l}{ingresos\_presencial} & Numeric & \multicolumn{1}{l|}{\begin{tabular}[c]{@{}l@{}}Ingresos por ventas presenciales de un local, en euros. \end{tabular}} \\ \hline
\multicolumn{1}{|l}{ingresos\_domicilio} & Numeric & \multicolumn{1}{l|}{\begin{tabular}[c]{@{}l@{}}Ingresos por ventas a domicilio de un local, en euros. \end{tabular}} \\ \hline
\multicolumn{1}{|l}{numero\_clientes\_presencial} & Integer & \multicolumn{1}{l|}{\begin{tabular}[c]{@{}l@{}}Número total de clientes presenciales de un local. \end{tabular}} \\ \hline
\multicolumn{1}{|l}{numero\_clientes\_domicilio} & Integer & \multicolumn{1}{l|}{\begin{tabular}[c]{@{}l@{}}Número total de clientes a domicilio de un local. \end{tabular}} \\ \hline
\multicolumn{1}{|l}{nuevos\_clientes\_presencial} & Integer & \multicolumn{1}{l|}{\begin{tabular}[c]{@{}l@{}}Número de clientes presenciales nuevos de un local. \end{tabular}} \\ \hline
\multicolumn{1}{|l}{nuevos\_clientes\_domicilio} & Integer & \multicolumn{1}{l|}{\begin{tabular}[c]{@{}l@{}}Número de clientes nuevos a domicilio de un local. \end{tabular}} \\ \hline
\multicolumn{1}{l}{} &  & \\ \hline
\end{tabular}
\end{table}

\begin{table}[H]
\centering
\begin{tabular}{cll}
\hline
\multicolumn{3}{|c|}{Producto} \\ \hline \hline
\multicolumn{1}{|c}{Atributo} & \multicolumn{1}{c}{Tipo} & \multicolumn{1}{c|}{Descripción} \\ \hline
\multicolumn{1}{|l}{ventas} & Integer & \multicolumn{1}{l|}{\begin{tabular}[c]{@{}l@{}}Número total de ventas de un producto en un local. \end{tabular}} \\ \hline
\multicolumn{1}{l}{} &  & \\ \hline
\multicolumn{3}{|c|}{Feedback} \\ \hline \hline
\multicolumn{1}{|c}{Atributo} & \multicolumn{1}{c}{Tipo} & \multicolumn{1}{c|}{Descripción} \\ \hline
\multicolumn{1}{|l}{valoracion\_ambiente} & Numeric & \multicolumn{1}{l|}{\begin{tabular}[c]{@{}l@{}}Valoración promedio (entre 0 y 5 con un decimal) \\ del ambiente de un local, según sus clientes. \end{tabular}} \\ \hline
\multicolumn{1}{|l}{valoracion\_personal} & Numeric & \multicolumn{1}{l|}{\begin{tabular}[c]{@{}l@{}}Valoración promedio (entre 0 y 5 con un decimal) \\ del personal de un local, según sus clientes. \end{tabular}} \\ \hline
\multicolumn{1}{|l}{valoracion\_comida} & Numeric & \multicolumn{1}{l|}{\begin{tabular}[c]{@{}l@{}}Valoración promedio (entre 0 y 5 con un decimal) \\ de la calidad de la comida de un local, según sus clientes. \end{tabular}} \\ \hline
\end{tabular}
\caption{Borrar esto: Básicamente en cada tabla de estas habrá una fila por cada restaurante y por cada mes y año.}
\end{table}

\newpage 

\subsection{Creación de los cubos de datos}

Utilizando como base las estructuras de datos relacionales descritas en el apartado anterior, se crea un esquema multidimensional con tres cubos y tres dimensiones compartidas entre ellos. Estas se definen a nivel de esquema y posteriormente se reutilizan a nivel de cubo:

\begin{itemize}
\item Dimensión: \textbf{Tiempo}. Dimensión de tipo temporal.
\begin{itemize}
\item Jerarquía: \textit{jerarquiaTiempo}. Definida por el atributo \textit{id}. 
\begin{itemize}
\item Nivel: \textit{ano}. Nivel de tipo \textit{TimeYears} definido por el atributo \textit{ano}.
\item Nivel: \textit{mes}. Nivel de tipo \textit{TimeMonths} definido por el atributo \textit{mes}. Se utiliza el atributo \textit{mes\_texto} para nombrar a los elementos de este nivel.
\item Tabla: tiempo
\end{itemize}
\end{itemize}
\item Dimensión: \textbf{Restaurante}. Dimensión de tipo estándar.
\begin{itemize}
\item Jerarquía: \textit{jerarquiaRestaurantes}. Definida por el atributo \textit{id}.
\begin{itemize}
\item Nivel: \textit{pais}. Nivel de tipo regular definido por el atributo \textit{pais}.
\item Nivel: \textit{ciudad}. Nivel de tipo regular definido por el atributo \textit{ciudad}.
\item Tabla: restaurante
\end{itemize}
\end{itemize}
\item Dimensión: \textbf{Producto}. Dimensión de tipo estándar.
\begin{itemize}
\item Jerarquía: \textit{jerarquiaProductos}. Definida por el atributo \textit{id}.
\begin{itemize}
\item Nivel: \textit{nombre}. Nivel de tipo regular definido por el atributo \textit{nombre}.
\item Tabla: producto
\end{itemize}
\end{itemize}
\item Cubo: \textbf{Finanzas}. 
\begin{itemize}
\item Tabla: \textit{finanzas}
\item Dimensiones 
\begin{itemize}
\item Dimensión usada: \textbf{Tiempo}. Se referencia a la dimensión \textit{Tiempo}. Se utiliza el atributo \textit{fecha} como clave foránea.
\item Dimensión usada: \textbf{Restaurante}. Se referencia a la dimensión \textit{Restaurante}. Se utiliza el atributo \textit{restaurante} como clave foránea.
\end{itemize}
\item Medidas \textcolor{red}{MIRAR BIEN PORQUE NO SE AGREGAN ?}
\begin{itemize}
\item Medida: \textbf{Alquiler}. ??
\item Medida: \textbf{Personal}. ??
\item Medida: \textbf{Proveedores}. ??
\item Medida: \textbf{Extra}. ??
\item Medida: \textbf{Ingresos presencial}. ??
\item Medida: \textbf{Ingresos domicilio}. ??
\item Medida: \textbf{Número clientes presencial}. ??
\item Medida: \textbf{Número clientes domicilio}. ??
\item Medida: \textbf{Nuevos clientes presencial}. ??
\item Medida: \textbf{Nuevos clientes domicilio}. ??
\item Medida calculada: \textbf{Gastos totales}. Se genera un nuevo valor para la dimensión \textit{Measures} usando la expresión MDX: ``[Measures].[Alquiler] + [Measures].[Personal] + [Measures].[Proveedores] + [Measures].[Extra]''
\item Medida calculada: \textbf{Ingresos totales}. Se genera un nuevo valor para la dimensión \textit{Measures} usando la expresión MDX: ``[Measures].[Ingresos presencial] + [Measures].[Ingresos domicilio]''
\item Medida calculada: \textbf{Beneficio}. Se genera un nuevo valor para la dimensión \textit{Measures} usando la expresión MDX: ``[Measures].[Ingresos totales] - [Measures].[Gastos totales]''
\item Medida calculada: \textbf{Número total clientes}. Se genera un nuevo valor para la dimensión \textit{Measures} usando la expresión MDX: ``[Measures].[Número clientes presencial] + [Measures].[Número clientes domicilio]''
\item Medida calculada: \textbf{Número total nuevos clientes}. Se genera un nuevo valor para la dimensión \textit{Measures} usando la expresión MDX: ``[Measures].[Nuevos clientes] + [Measures].[Nuevos clientes domicilio]''
\item Medida calculada: \textbf{Beneficio por cliente}. Se genera un nuevo valor para la dimensión \textit{Measures} usando la expresión MDX: ``[Measures].[Beneficio] / [Measures].[Número total clientes]''
\end{itemize}
\end{itemize}
\item Cubo: \textbf{Producto}.
\begin{itemize}
\item Tabla: \textit{producto}
\item Dimensiones
\begin{itemize}
\item Dimensión usada: \textbf{Tiempo}. Se referencia a la dimensión \textit{Tiempo}. Se utiliza el atributo \textit{fecha} como clave foránea.
\item Dimensión usada: \textbf{Restaurante}. Se referencia a la dimensión \textit{Restaurante}. Se utiliza el atributo \textit{restaurante} como clave foránea.
\item Dimensión usada: \textbf{Producto}. Se referencia a la dimensión \textit{Producto}. Se utiliza el atributo \textit{producto} como clave foránea.
\end{itemize}
\item Medidas \textcolor{red}{como trato a precio? lo meto en la dimension producto?}
\begin{itemize}
\item Medida: \textbf{Ventas}. ??
\item Medida calculada: \textbf{Ingresos por producto}. Se genera un nuevo valor para la dimensión \textit{Measures} usando la expresión MDX: ``[Measures].[Ventas] * [Producto].[precio]''
\end{itemize}
\end{itemize}
\item Cubo: \textbf{Feedback}.
\begin{itemize}
\item Tabla: \textit{feedback}
\item Dimensiones
\begin{itemize}
\item Dimensión usada: \textbf{Tiempo}. Se referencia a la dimensión \textit{Tiempo}. Se utiliza el atributo \textit{fecha} como clave foránea.
\item Dimensión usada: \textbf{Restaurante}. Se referencia a la dimensión \textit{Restaurante}. Se utiliza el atributo \textit{restaurante} como clave foránea.
\end{itemize}
\item Medidas
\begin{itemize}
\item Medida: \textbf{Valoración ambiente}. Se agrega el atributo \textit{valoracion\_ambiente} usando la funcion AVG.
\item Medida: \textbf{Valoración personal}. Se agrega el atributo \textit{valoracion\_personal} usando la funcion AVG.
\item Medida: \textbf{Valoración comida}. Se agrega el atributo \textit{valoracion\_comida} usando la funcion AVG.
\item Medida calculada: \textbf{Valoración media restaurante}. Se genera un nuevo valor para la dimensión \textit{Measures} usando la expresión MDX: ``([Measures].[Valoración ambiente] + [Measures].[Valoración personal] + [Measures].[Valoración comida]) / 3''
\end{itemize}
\end{itemize}
\end{itemize}

\singlespacing
\end{document}

