\documentclass{article}
\usepackage{tikz}
\usepackage[left=0.15cm, right=2cm, top=0.5cm, bottom=0.5cm]{geometry}
\usepackage{xcolor}
\usepackage{setspace}
\usepackage{enumitem}
\usepackage{multicol}

% Definir colores personalizados
\definecolor{lightyellow}{HTML}{FFF9C4}
\definecolor{softpeach}{HTML}{FFE0B2}
\definecolor{palegreen}{HTML}{DCEDC8}
\definecolor{skyblue}{HTML}{BBDEFB}
\definecolor{lavender}{HTML}{E1BEE7}
\definecolor{mint}{HTML}{B2DFDB}
\definecolor{lightgray}{HTML}{F5F5F5}
\definecolor{pastelpink}{HTML}{F8BBD0}
\definecolor{powderblue}{HTML}{B3E5FC}
\definecolor{lemoncream}{HTML}{FFF8E1}


\begin{document}

\begin{scriptsize}
\begin{tikzpicture}
    \node[fill=lemoncream, fill opacity=1, text opacity=1, text width=0.48\textwidth, align=justify, draw, rounded corners, inner sep=10pt] 
    (note1) at (-3.85, 0.7) {
        \textbf{\textsf{\textcolor{red}{S}AY}. Transmitir mensaje claro y conciso} \\[-10pt]
        \begin{itemize}[topsep=2pt, itemsep=0pt]
        \item[SA1.] Conoce objetivos.
        \begin{itemize}[topsep=-3pt, itemsep=-2pt]
        \item[SA1.1] Conoce tus metas.
        \item[SA1.2] Conoce tu audiencia.
        \end{itemize}
        \item[SA2.] Introduce el mensaje.
        \begin{itemize}[topsep=-3pt, itemsep=-2pt]
        \item[SA2.1] Presenta hechos relacionados.
        \item[SA2.2] Explica el problema.
        \item[SA2.3] ¿Qué es relevantes desde la perpectiva del usuario?
        \end{itemize}
        \item[SA3.] Entrega el mensaje.
        \begin{itemize}[topsep=-3pt, itemsep=-2pt]
        \item[SA3.1] Detecta, explica y sugiere.
        \item[SA3.2] Di primero el mensaje.
        \end{itemize}
        \item[SA4.] Apoya el mensaje.
        \begin{itemize}[topsep=-3pt, itemsep=-2pt]
        \item[SA4.1] Da evidencias.
        \item[SA4.2] Usa palabras precisas.
        \item[SA4.3] Resalta el mensaje.
        \item[SA4.4] Nombra las fuentes.
        \item[SA4.5] Enlaza comentarios.
        \end{itemize}
        \item[SA5.] Resume el mensaje.
        \begin{itemize}[topsep=-3pt, itemsep=-2pt]
        \item[SA5.1] Repite el mensaje.
        \item[SA5.2] Explica las conscuencias.
        \end{itemize}
        \end{itemize}
    };
    \node[fill=lemoncream, fill opacity=1, text opacity=1, text width=0.48\textwidth, align=justify, draw, rounded corners, inner sep=10pt] 
    (note1) at (6.4, 0.02) {
        \textbf{\textsf{\textcolor{red}{U}NIFY}. Coherencia semántica a contenido similar.} \\[-10pt]
        \begin{itemize}[topsep=2pt, itemsep=0pt]
        \item[UN1.] Unificar terminología.
        \begin{itemize}[topsep=-3pt, itemsep=-2pt]
        \item[UN1.1] Unificar términos y abreviaciones.
        \item[UN1.2] Unificar números, unidades, fechas.
        \end{itemize}
        \item[UN2.] Unificar descriptores
        \begin{itemize}[topsep=-3pt, itemsep=-2pt]
        \item[UN2.1] Unificar mensajes.
        \item[UN2.2] Unificar títulos y subtítulos.
        \item[UN2.3] Unificar posiciones de leyenda y etiquetas.
        \item[UN2.4] Unificar comentarios.
        \item[UN2.5] Unificar pies de página.
        \end{itemize}
        \item[UN3.] Unificar dimensiones.
        \begin{itemize}[topsep=-3pt, itemsep=-2pt]
        \item[UN3.1] Unificar medidas (basic 2/3 ancho, ratio 1/3 ancho).
        \item[UN3.2] Unificar escenarios.
        \item[UN3.3] Unificar periodos temporales (usar eje horiz. direc. visual del tiempo).
        \item[UN3.4] Unificar estructura de las dimensiones.
        \end{itemize}
        \item[UN4.] Unificar análisis.
        \begin{itemize}[topsep=-3pt, itemsep=-2pt]
        \item[UN4.1] Unificar escenarios de análisis.
        \item[UN4.2] Unificar análisis de series temporales.
        \item[UN4.3] Unificar estructura de análisis.
        \item[UN4.4] Unificar análisis de ajuste.
        \end{itemize}
        \item[UN5.] Unificar indicadores
        \begin{itemize}[topsep=-3pt, itemsep=-2pt]
        \item[UN5.1] Unificar indicadores de resalte.
        \item[UN5.2] Unificar indicadores de escala.
        \item[UN5.3] Unificar indicadores de \textit{outliers}
        \end{itemize}
        \end{itemize}
    };
    \node[fill=lemoncream, fill opacity=1, text opacity=1, text width=0.48\textwidth, align=justify, draw, rounded corners, inner sep=10pt] 
    (note1) at (-3.85, -6.65) {
        \textbf{\textsf{\textcolor{red}{C}ONDENSE}. Aumentar densidad de información manteniendo claridad} \\[-10pt]
        \begin{itemize}[topsep=2pt, itemsep=0pt]
        \item[CO1.] Usar componentes pequeños.
        \begin{itemize}[topsep=-3pt, itemsep=-2pt]
        \item[CO1.1] Fuentes pequeñas.
        \item[CO1.2] Elementos pequeños.
        \item[CO1.3] Objetos pequeños.
        \end{itemize}
        \item[CO2.] Maximizar el uso de espacio.
        \begin{itemize}[topsep=-3pt, itemsep=-2pt]
        \item[CO2.1] Márgenes estrechos.
        \item[CO2.2] Reducir espacios blancos.
        \end{itemize}
        \item[CO3.] Añadir datos.
        \begin{itemize}[topsep=-3pt, itemsep=-2pt]
        \item[CO3.1] Añadir puntos de datos.
        \item[CO3.2] Añadir dimensiones.
        \end{itemize}
        \item[CO4.] Añadir elementos. 
        \begin{itemize}[topsep=-3pt, itemsep=-2pt]
        \item[CO4.1] Gráficos superpuestos.
        \item[CO4.2] Gráficos de varias categorías.
        \item[CO4.3] Gráficos extendidos.
        \item[CO4.4] Gráficos embebidos en tablas.
        \item[CO4.5] Explicaciones embebidas en gráficos.
        \end{itemize}
        \item[CO5.] Añadir objetos.
        \begin{itemize}[topsep=-3pt, itemsep=-2pt]
        \item[CO5.1] Gráficos relacionados en una página.
        \item[CO5.2] Combinar tablas y gráficos.
        \item[CO5.3] Tablas y gráficos en páginas de texto.
        \end{itemize}
        \end{itemize}
    };
    \node[fill=lemoncream, fill opacity=1, text opacity=1, text width=0.48\textwidth, align=justify, draw, rounded corners, inner sep=10pt] 
    (note1) at (6.4, -7.48) {
        \textbf{\textsf{\textcolor{red}{C}HECK}. Información comprensible sin malinterpretaciones.} \\[-15pt]
        \begin{multicols}{2}
        \begin{itemize}[topsep=2pt, itemsep=0pt]
        \item[CH1.] Evitar ejes manipulados.
        \begin{itemize}[topsep=-3pt, itemsep=-2pt]
        \item[CH1.1] Evitar ejes truncados.
        \item[CH1.2] Evitar ejes logarítmicos.
        \item[CH1.3] Evitar distintos tamaños de clases
        \end{itemize}
        \item[CH2.] Evitar elementos manipulados.
        \begin{itemize}[topsep=-3pt, itemsep=-2pt]
        \item[CH2.1] Evitar datos cortados.
        \item[CH2.2] Soluciones creativas para el escalado.
        \end{itemize}
        \item[CH3.] Evitar representaciones malinterpretables.
        \begin{itemize}[topsep=-3pt, itemsep=-2pt]
        \item[CH3.1] Relación entre áreas correcta (mejor usar escalas lineales).
        \item[CH3.2] Relación entre volúmenes correcta.
        \item[CH3.3] Evitar colores malinterp. en mapas.
        \end{itemize} \columnbreak
        \item[CH4.] Usar mismas escalas.
        \begin{itemize}[topsep=-3pt, itemsep=-2pt]
        \item[CH4.1] Misma escala para mismas unidades.
        \item[CH4.2] Ajustar tamaño del gráfico a los datos.
        \item[CH4.3] Usar indicadores de escala.
        \item[CH4.4] Usar indicadores de \textit{outliers}.
        \item[CH4.5] Usar lupas amplificadores.
        \end{itemize}
        \item[CH5.] Mostrar ajustes de los datos.
        \begin{itemize}[topsep=-3pt, itemsep=-2pt]
        \item[CH5.1] Mostrar impacto de la inflación.
        \item[CH5.2] Mostrar impacto de la moneda.
        \end{itemize}
        \end{itemize}
    \end{multicols}
    };
    \node[fill=lemoncream, fill opacity=1, text opacity=1, text width=0.48\textwidth, align=justify, draw, rounded corners, inner sep=10pt] 
    (note1) at (-3.85, -13.25) {
        \textbf{\textsf{\textcolor{red}{E}XPRESS}. Visualización adecuada.} \\[-15pt]
        \begin{multicols}{2}
        \begin{itemize}[topsep=2pt, itemsep=0pt]
        \item[EX1.] Usa el tipo de objetos adecuado.
        \begin{itemize}[topsep=-3pt, itemsep=-2pt]
        \item[EX1.1] Tipo de gráfico adecuado.
        \item[EX1.2] Tipo de tabla adecuado.
        \end{itemize}
        \item[EX2.] Reemplaza gráficos inadecuados.
        \begin{itemize}[topsep=-3pt, itemsep=-2pt]
        \item[EX2.1] \textit{Pie} \& \textit{ring charts}.
        \item[EX2.2] \textit{Gauges} \& \textit{speedometers}.
        \item[EX2.3] \textit{Radar} \& \textit{funnel charts}.
        \item[EX2.4] \textit{Spaguetti charts}.
        \item[EX2.5] \textit{Traffic lights}.
        \end{itemize}
        \item[EX3.] Reemplaza representaciones inadecuadas.
        \begin{itemize}[topsep=-3pt, itemsep=-2pt]
        \item[EX3.1] Mejor representaciones cuantitativas.
        \item[EX3.2] Evitar diapositivas con texto.
        \end{itemize}
        \item[EX4.] Añade comparaciones.
        \begin{itemize}[topsep=-3pt, itemsep=-2pt]
        \item[EX4.1] Entre escenarios.
        \item[EX4.2] Varianzas.
        \end{itemize}
        \item[EX5.] Explica las causas.
        \begin{itemize}[topsep=-3pt, itemsep=-2pt]
        \item[EX5.1] Muestra estructuras de árbol.
        \item[EX5.2] Muestra \textit{clusters}.
        \item[EX5.3] Muestra correlaciones.
        \end{itemize}
        \end{itemize}
    \end{multicols}
    };
    \node[fill=lemoncream, fill opacity=1, text opacity=1, text width=0.48\textwidth, align=justify, draw, rounded corners, inner sep=10pt] 
    (note1) at (6.4, -14.3) {
    \textbf{\textsf{\textcolor{red}{S}IMPLIFY}. Quitar cosas que no den valor.} \\[-10pt]
        \begin{itemize}[topsep=2pt, itemsep=0pt]
        \item[SI1.] Evita componentes innecesarios.
        \begin{itemize}[topsep=-3pt, itemsep=-2pt]
        \item[SI1.1] Evita layouts desordenados.
        \item[SI1.2] Evita fondos coloreados.
        \item[SI1.3] Evita animaciones/transiciones.
        \end{itemize}
        \item[SI2.] Evita estilos decorativos.
        \begin{itemize}[topsep=-3pt, itemsep=-2pt]
        \item[SI2.1] Evita marcos, sombreados, pseudo-3D.
        \item[SI2.2] Evita colores decorativos.
        \item[SI2.3] Evita fuentes decorativas.
        \end{itemize}
        \item[SI3.] Reemplaza por un layout limpio.
        \begin{itemize}[topsep=-3pt, itemsep=-2pt]
        \item[SI3.1] Etiquetas en vez de grid o ejes etiquetados.
        \item[SI3.2] Alinear a la derecha en vez de líneas verticales en tablas.
        \end{itemize}
        \item[SI4.] Evita redundancia. 
        \begin{itemize}[topsep=-3pt, itemsep=-2pt]
        \item[SI4.1] Evita palabras extra.
        \item[SI4.2] Evita términos obvios.
        \item[SI4.3] Evita repetir palabras.
        \end{itemize}
        \item[SI5.] Evita detalles que distraigan
        \begin{itemize}[topsep=-3pt, itemsep=-2pt]
        \item[SI5.1] Evita etiquetas para valores pequeños.
        \item[SI5.2] Evita números largos.
        \item[SI5.3] Evita etiquetas innecesarias.
        \end{itemize}
        \end{itemize}
    };
    \node[fill=lemoncream, fill opacity=1, text opacity=1, text width=\textwidth, align=justify, draw, rounded corners, inner sep=10pt] 
    (note1) at (1.35, -20) {
        \textbf{\textsf{\textcolor{red}{S}TRUCTURE}. Estructura lógica, que cuente una historia.} \\[-15pt]
        \begin{multicols}{2}
        \begin{itemize}[topsep=2pt, itemsep=0pt]
        \item[ST1.] Usa elementos consistentes (cons).
        \begin{itemize}[topsep=-3pt, itemsep=-2pt]
        \item[ST1.1] Items consistentes.
        \item[ST1.2] Tipos de afirmaciones consistentes.
        \item[ST1.3] Palabras consistentes.
        \item[ST1.4] Visualizaciones consistentes.
        \end{itemize}
        \item[ST2.] Evitar elementos superpuestos.
        \begin{itemize}[topsep=-3pt, itemsep=-2pt]
        \item[ST2.1] \textit{No-overlapping} estructuras de reportes.
        \item[ST2.2] \textit{No-overlapping} medidas de negocio.
        \item[ST2.3] \textit{No-overlapping} estructura de dimensiones.
        \end{itemize} \columnbreak
        \item[ST3.] Construye elementos exhaustivos.
        \begin{itemize}[topsep=-3pt, itemsep=-2pt]
        \item[ST3.1] Argumentos exhaustivos.
        \item[ST3.2] Estructuras exhaustivos.
        \end{itemize}
        \item[ST4.] Construye estructuras jerárquicas.
        \begin{itemize}[topsep=-3pt, itemsep=-2pt]
        \item[ST4.1] Razonamiento deductivo.
        \item[ST4.2] Razonamiento inductivo.
        \end{itemize}
        \item[ST5.] Estructura visual.
        \begin{itemize}[topsep=-3pt, itemsep=-2pt]
        \item[ST5.1] Estructura visual en reporte.
        \item[ST5.2] Estructura visual en tablas.
        \item[ST5.3] Estructura visual en notas.
        \end{itemize}
        \end{itemize}
        \end{multicols}
    };
\end{tikzpicture}
\end{scriptsize}

\end{document}
